
\noindent {\color{special}{\Large \bf LIÇÃO 2 - Para o professor}}
\vspace{.5cm}

Na Lição anterior, as frações unitárias permitiram reconhecer e registrar quantidades menores que a unidade: meio, um terço, um quarto, etc. Nesta, serão abordadas as frações em geral: as que representam quantidades menores do que a unidade, quantidades maiores do que a unidade ou quantidades inteiras. Também serão introduzidas a notação simbólica de frações e a comparação entre frações de mesmo denominador. 

As frações com numerador diferente de 1 são apresentadas \textbf{reunindo-se (por contagem ou por justaposição) cópias de uma mesma fração unitária.}
Cabe ressaltar que reunir aqui não tem como objetivo tratar da operação de adição, ou seja, não se espera que o aluno compreenda, nem que seja apresentado a ele, $2/3$ como $1/3 + 1/3$. Espera-se no entanto que o raciocínio aditivo, elementar na contagem, ampare a compreensão do aluno.

Para isso, tem-se a representação pictórica como um apoio importante.
Na Atividade 1, as imagens das pizzas amparam a compreensão das frações três quartos, cinco sextos e cinco oitavos como a reunião de cópias de frações unitárias correspondentes à quarta parte, à sexta parte e à oitava parte de uma pizza, respectivamente. Nesse sentido, nas primeiras atividades, há um esforço deliberado para que o estudante faça uso da linguagem de frações apresentada na Lição 1 para expressar frações não unitárias.
Na Atividade 2, as imagens da barra de chocolate amparam a compreensão da fração dois terços como a reunião de duas partes correspondentes à terça parte (ou à fração um terço) de uma barra de chocolate.
Na Atividade 3, sabendo que uma das cinco fatias iguais em que foi repartida uma torta é um quinto da torta, espera-se que o aluno use a linguagem ``dois quintos'' ou ``dois um quintos'' da torta para se referir às outras duas fatias.
Dessa forma, ``dois quintos'' da torta são obtidos pela reunião de duas partes correspondentes a ``um quinto'' da torta.
O objetivo é que esse processo se estenda para a compreensão das frações não unitárias.
Assim, as frações ``quatro quintos'' e ``seis quintos'' são entendidas como a reunião de ``quatro um quintos'' e ``seis um quintos'' da mesma unidade, respectivamente. O processo de reunir frações unitárias pode ser observado como uma contagem (com justaposição ou não) em que a fração unitária tem o papel de unidade na contagem. A fração unitária é uma subunidade da unidade considerada. 

Um cuidado especial recomendado ao professor é com as frações que representam uma quantidade maior que a unidade, introduzidas logo nas primeiras atividades ainda sem notação simbólica. Não é indicado atrasar muito a introdução deste tipo de fração porque o estudante pode fixar-se na ideia de que não há fração maior do que a unidade  (por exemplo, a fração quatro terços pode não fazer sentido para o estudante porque, para ele, não faz sentido dividir uma torta em 3 pedaços e tomar 4). Decidiu-se omitir as terminologias ``fração própria'', ``fração imprópria'' e ``fração aparente'' por se acreditar que esta linguagem não só é desnecessária para a aprendizagem do assunto como também por poder desviar a atenção dos conceitos que realmente importam.
A representação de frações maiores que a unidade na forma de número misto também não é adotada neste texto. Esse tipo de representação não tem um uso que justifique a abordagem nesta etapa da aprendizagem.

Apesar de esta lição introduzir a linguagem simbólica de frações, o estudante talvez ainda precise de uma unidade concreta explícita para dar um significado para a fração $\frac{a}{b}$: por exemplo, ``$\frac{a}{b}$ de uma pizza'' ou ``$\frac{a}{b}$ de uma barra de chocolate''. Apenas na próxima lição, a fração $\frac{a}{b}$ será tratada como número, requerendo a abstração que o conceito de número exige.
\vspace{.15cm}

\noindent OBJETIVOS ESPECÍFICOS DA LIÇÃO 2:
\vspace{.15cm}

\noindent O aluno deve ser capaz de:
\begin{itemize}
 \item  Reconhecer a necessidade de apresentar uma expressão verbal que identifique a quantidade correspondente à junção de duas ou mais partes correspondentes às frações unitárias de mesmo denominador.
 \item  Reconhecer frações não unitárias como a reunião de cópias de frações unitárias de uma mesma unidade. 
 \item  Utilizar as linguagens verbal e simbólica para referir-se a uma fração $\frac{a}{b}$.
 \item  Reconhecer e nomear os termos de uma fração (numerador e denominador). 
 \item  Comparar frações de mesmo denominador.
 \item Reconhecer o papel da unidade na identificação da fração, tanto em situações em  que uma mesma fração pode representar quantidades diferentes como em situações em que frações diferentes podem representar uma mesma quantidade.
\end{itemize}


\pagebreak

%\begin{multicols}{2}
\begin{objetivos}[code={\setcounter{tcb@cnt@objetivos}{0}}, label=chap2-ativ1]{}{}  %o comando entre colchetes reinicia a numeração dos objetivos manualmente do zero.
  \begin{itemize} %s
\item Estender o conceito de frações para expressar quantidades que correspondam a mais do que uma fração unitária, a partir da junção de duas ou mais partes correspondentes às frações unitárias de mesmo denominador.
\item Reconhecer a necessidade de apresentar uma expressão verbal que identifique a quantidade correspondente à junção de duas ou mais partes correspondentes às frações unitárias de mesmo denominador.
  \end{itemize} %s
\end{objetivos}

\begin{orientacoes}{}{}
  \begin{itemize} %s
\item Recomenda-se que a atividade seja realizada individualmente e que os alunos compartilhem suas respostas. 
\item É possível que os alunos utilizem expressões variadas para nomear as quantidades de pizza que cada amigo comeu. Por exemplo, ``três de quatro fatias'', ``três fatias de um quarto de pizza'', ``três quartos de pizza'', dentre outras. É importante que a discussão conduza os alunos ao uso de quartos, sextos e oitavos: ``três quartos'', ``cinco sextos'', ``cinco oitavos'', etc. 
\end{itemize} %s
\end{orientacoes}
%  \noindent {\bf Classificações:}\vspace{.1cm}
% \begin{itemize} %s
%     \item       Heid et al.: Conceito: identificar e descrever
%     \item       Nicely, Jr.: Nível 1: reconhecer
%     \item       UERJ: Observar: identificar e reconhecer
% \end{itemize} %s

\begin{solucao}[code={\setcounter{tcb@cnt@solucao}{0}}]{}{}
  \label{ativ:sobra-de-pizza}
\begin{enumerate} [\quad a)] %d
\item Da esquerda para a direita as pizzas são de João, Mariele e Luiza.
\item Na pizza de João uma fatia representa um sexto da pizza. \newline
  Na pizza de Mariele uma fatia representa um oitavo da pizza. \newline
  Na pizza de Luiza uma fatia representa um quarto da pizza.
\item João comeu quatro sextos de sua pizza; Mariele, seis oitavos; e Luiza comeu três quartos de sua pizza.
\end{enumerate} %d
\end{solucao}
\vfill

  \begin{objetivos}[label=chap2-ativ2]{}{}
  \begin{itemize} %s
    % \item Estender o uso de frações para expressar quantidades que correspondam a mais do que uma fração unitária, a partir da justaposição de duas ou mais partes correspondentes às frações unitárias de mesmo denominador.
    % \item Reconhecer e usar frações para expressar quantidades que correspondam a mais do que uma fração unitária, em situação de equipartição de mais do que uma unidade (no caso, duas).
    \item Reconhecer a necessidade de apresentar uma expressão verbal que identifique a quantidade correspondente à junção de duas ou mais partes correspondentes às frações unitárias de mesmo denominador.
    % \item Compreender e usar a expressão ``$n$ terços de''como forma de registrar as $n$ partes da equipartição da unidade em três partes (no caso, dois terços).
    % \item Identificar a fração ``$n$ terços de'' em uma situação de equipartição de mais do que uma unidade.
\item Compreender e usar as expressões ``dois terços'', ``três terços'' e ``quatro terços'' como forma de registrar as duas, três ou quatro partes de uma equipartição da unidade em três partes.
  \end{itemize} %s
\end{objetivos}
\newpage

\begin{orientacoes}{}{}
\begin{itemize} %s
   \item Recomenda-se que a atividade seja desenvolvida em grupos de 3 a 5 alunos.
   \item Nesta atividade, é importante que os alunos possam ter cópias de figuras ilustrativas das barras de chocolate para dividir e poder avaliar e decidir as suas respostas. Faça cópias das páginas para reprodução.
   \item A opção por um problema de divisão em partes iguais (dois terços corresponde ao resultado da divisão de 2 unidades em 3 partes iguais) no lugar da abordagem ``parte-todo'' (dois terços corresponde a duas partes da equipartição da unidade em três partes) que, no Brasil, costuma ser o mais tradicional, tem duas justificativas: (1) manter a abordagem que marca a lição anterior: equipartição (agora com múltiplas cópias da unidade); (2) amparar a compreensão de frações cujo numerador é maior do que o denominador pela reunião de cópias de partes da unidade (uma vez que pode não parecer coerente nomear uma parte ``maior do que o todo'').
   \item As diversas soluções apresentadas pelos diferentes grupos devem ser discutidas com a turma inteira.
   \item É possível que os alunos utilizem expressões variadas para nomear as partes dos chocolates em cada divisão e para a quantidade de chocolate que cada irmão recebeu. Por exemplo, ``dois dos seis pedaços'', ``dois pedaços de um terço de chocolate''. É importante que a discussão conduza os alunos ao uso de terços:       ``dois terços'', ``quatro terços'', ``seis terços'', etc. Observa-se que o uso de ``sextos'' muito provavelmente indica uma confusão do aluno em relação ao reconhecimento da unidade.
  
\end{itemize} %s

%  \noindent {\bf Classificações:}\vspace{.1cm}
% \begin{itemize} %s
%     \item       Heid et al.: Conceito: identificar e descrever
%     \item       Nicely, Jr.: Nível 1: reconhecer
%     \item       UERJ: Observar: identificar e reconhecer
% \end{itemize} %s
\end{orientacoes}

\begin{solucao}{}{}
\begin{enumerate} [\quad a)] %d
    \item  Um terço.
    \item  Sim, pois a divisão foi justa no sentido de cada irmã ter recebido a mesma quantidade de chocolate.
    \item  Sim, pois cada irmã recebeu dois pedaços que equivalem, cada um, a um terço de uma barra de chocolate.
    \item  Dois terços de uma barra.
    \item  Três terços de uma barra, ou seja, uma barra inteira de chocolate.
    \item  Quatro terços de uma barra, ou seja, uma barra inteira e um terço de chocolate.
\end{enumerate} %d

\end{solucao}

\newpage

\begin{multicols}{2}
\begin{objetivos}[label=chap2-ativ3]{}{}
  \begin{itemize} %s
\item Reconhecer a necessidade de apresentar uma expressão verbal que identifique a quantidades correspondentes a $n$ quintos.
\item Compreender e usar a expressão ``$n$ quintos de'' como uma forma de identificar a quantidade equivalente a $n$ partes da equipartição da unidade em quintos, incluindo os casos em que $n$ é maior do que cinco.
\item Comparar frações de mesmo denominador em uma situação.
\end{itemize} %s
\end{objetivos}

\begin{orientacoes}{}{}
  \begin{itemize} %s
  \item Recomenda-se que a atividade seja desenvolvida em grupos de 3 a 5 alunos.
  \item Nesta atividade, é importante que os alunos possam ter cópias de figuras ilustrativas da torta para dividir e poder avaliar e decidir suas respostas. Faça cópias das páginas para reprodução e entregue uma para cada grupo.
  \item       As diversas soluções apresentadas pelos diferentes grupos devem ser discutidas com a turma inteira.
  \item       Em particular, no Item a), não se espera, nem se recomenda, que a representação feita pelos alunos seja amparada por medida. O objetivo é que façam a equipartição livremente e de forma coerente. Assim, por exemplo, pode ser aceita como resposta a solução indicada na figura a seguir.


\begin{center}
 \begin{tikzpicture}[yscale=.45,xscale=.4, x=1mm,y=1mm, rotate=90]
 \draw (0,0) rectangle (30,60);
 \foreach \x in {12,24,...,48} \draw (0,\x) -- (30,\x);
 \node[rotate=90, attention] at (15,54) {{\small Amarildo}};
 \node[rotate=90, attention] at (15,42) {{\small Beto}};
 \node[rotate=90, attention] at (15,30) {{\small Carlos}};
 \node[rotate=90, attention] at (15,18) {{\small Davi}};
 \node[rotate=90, attention] at (15,6) {{\small Edison}};
\end{tikzpicture}

\begin{tikzpicture}[yscale=.45,xscale=.4, x=1mm,y=1mm, rotate=90]
 \draw (0,0) rectangle (30,60);
 \foreach \x in {12,24,...,48} \draw (0,\x) -- (30,\x);
 \node[rotate=90, attention] at (15,54) {{\small Amarildo}};
 \node[rotate=90, attention] at (15,42) {{\small Beto}};
 \node[rotate=90, attention] at (15,30) {{\small Carlos}};
 \node[rotate=90, attention] at (15,18) {{\small Davi}};
 \node[rotate=90, attention] at (15,6) {{\small Edison}};
\end{tikzpicture}

\begin{tikzpicture}[yscale=.45,xscale=.4, x=1mm,y=1mm, rotate=90]
 \draw (0,0) rectangle (30,60);
 \foreach \x in {12,24,...,48} \draw (0,\x) -- (30,\x);
 \node[rotate=90, attention] at (15,54) {{\small Amarildo}};
 \node[rotate=90, attention] at (15,42) {{\small Beto}};
 \node[rotate=90, attention] at (15,30) {{\small Carlos}};
 \node[rotate=90, attention] at (15,18) {{\small Davi}};
 \node[rotate=90, attention] at (15,6) {{\small Edison}};
\end{tikzpicture}
\end{center}


    \item       Em suas respostas, é possível que os alunos utilizem expressões variadas para nomear as partes das tortas em cada divisão e para as quantidades de torta que cada irmão recebe. Por exemplo,       ``três dos quinze pedaços'',       ``três pedaços de um quinto de torta'', dentre outras. É importante que a discussão conduza os alunos ao uso de quintos:       ``três quintos'',       ``seis quintos'',       ``quinze quintos'', etc.
    \item       Espera-se que, ao final da atividade, o aluno reconheça o significado das expressões dois quintos e três quintos, mesmo que não o faça espontaneamente (usando, por exemplo, especificações como       ``dois pedaços''     ou       ``duas fatias'') e seja necessária a intervenção do professor. {\bf O professor deve fazer e incentivar o uso da terminologia de frações que se quer estabelecer nesta lição.}
   % \item Cabe ressaltar que no Item b) (II) já está implícita a adição, na medida em que é esperado do estudante juntar duas frações não unitárias. No entanto, não é recomendável aqui que se chame atenção do estudante para a adição ou que seja utilizada a notação ``+'', mas sim que se enfatize o conceito de fração não unitária: juntou-se, afinal, 6 pedaços iguais a um quinto de torta.
    \item       Nos Itens c) e d), não basta uma resposta       ``Sim''     ou       ``Não''. É importante estimular os seus alunos a darem uma justificativa.
\end{itemize} %s


%   \noindent {\bf Classificações:}\vspace{.1cm}
%   Itens a) e b) 
% \begin{itemize} %s
%     \item       Heid et al.: Conceito: identificar, descrever
%     \item       Nicely, Jr.: Nível 1: reconhecer
%     \item       UERJ: Observar: identificar, reconhecer
% \end{itemize} %s

%   Itens c) e d)
% \begin{itemize} %s
%     \item       Heid et al.: Raciocínio: justificar
%     \item       Nicely, Jr.: Nível 6: justificar
%     \item       UERJ: Avaliar: julgar
% \end{itemize} %s
\end{orientacoes}

\begin{solucao}{}{}
\begin{enumerate} [\quad a)] %d
    \item       Uma resposta possível: dividir cada uma das três tortas em 5 partes iguais e, então, com as 15 partes disponíveis, distribuir 3 partes para cada amigo, como mostra a figura a seguir

\begin{center}
 \begin{tikzpicture}[yscale=.45,xscale=.4, x=1mm,y=1mm, rotate=90]
 \draw (0,0) rectangle (30,60);
 \foreach \x in {12,24,...,48} \draw (0,\x) -- (30,\x);
 \node[rotate=90, attention] at (15,54) {{\small Amarildo}};
 \node[rotate=90, attention] at (15,42) {{\small Beto}};
 \node[rotate=90, attention] at (15,30) {{\small Carlos}};
 \node[rotate=90, attention] at (15,18) {{\small Davi}};
 \node[rotate=90, attention] at (15,6) {{\small Edison}};
\end{tikzpicture}\hspace{1em}
\begin{tikzpicture}[yscale=.45,xscale=.4, x=1mm,y=1mm, rotate=90]
 \draw (0,0) rectangle (30,60);
 \foreach \x in {12,24,...,48} \draw (0,\x) -- (30,\x);
 \node[rotate=90, attention] at (15,54) {{\small Amarildo}};
 \node[rotate=90, attention] at (15,42) {{\small Beto}};
 \node[rotate=90, attention] at (15,30) {{\small Carlos}};
 \node[rotate=90, attention] at (15,18) {{\small Davi}};
 \node[rotate=90, attention] at (15,6) {{\small Edison}};
\end{tikzpicture}\hspace{1em}
\begin{tikzpicture}[yscale=.45,xscale=.4, x=1mm,y=1mm, rotate=90]
 \draw (0,0) rectangle (30,60);
 \foreach \x in {12,24,...,48} \draw (0,\x) -- (30,\x);
 \node[rotate=90, attention] at (15,54) {{\small Amarildo}};
 \node[rotate=90, attention] at (15,42) {{\small Beto}};
 \node[rotate=90, attention] at (15,30) {{\small Carlos}};
 \node[rotate=90, attention] at (15,18) {{\small Davi}};
 \node[rotate=90, attention] at (15,6) {{\small Edison}};
\end{tikzpicture}
\end{center}

Outra resposta possível: dividir cada uma das três tortas em cinco partes iguais e distribuir três partes \textit{consecutivas} para cada amigo.

\begin{center}
 \begin{tikzpicture}[yscale=.45,xscale=.4, x=1mm,y=1mm, rotate=90]
 \draw (0,0) rectangle (30,60);
 \foreach \x in {12,24,...,48} \draw (0,\x) -- (30,\x);
 \node[rotate=90, attention] at (15,54) {{\small Amarildo}};
 \node[rotate=90, attention] at (15,42) {{\small Amarildo}};
 \node[rotate=90, attention] at (15,30) {{\small Amarildo}};
 \node[rotate=90, attention] at (15,18) {{\small Beto}};
 \node[rotate=90, attention] at (15,6) {{\small Beto}};
\end{tikzpicture}\hspace{1em}
\begin{tikzpicture}[yscale=.45,xscale=.4, x=1mm,y=1mm, rotate=90]
 \draw (0,0) rectangle (30,60);
 \foreach \x in {12,24,...,48} \draw (0,\x) -- (30,\x);
 \node[rotate=90, attention] at (15,54) {{\small Beto}};
 \node[rotate=90, attention] at (15,42) {{\small Carlos}};
 \node[rotate=90, attention] at (15,30) {{\small Carlos}};
 \node[rotate=90, attention] at (15,18) {{\small Carlos}};
 \node[rotate=90, attention] at (15,6) {{\small Davi}};
 \node[rotate=90, attention] at (15,6) {{\small\phantom{Amarildo}}};
\end{tikzpicture}\hspace{1em}
\begin{tikzpicture}[yscale=.45,xscale=.4, x=1mm,y=1mm, rotate=90]
 \draw (0,0) rectangle (30,60);
 \foreach \x in {12,24,...,48} \draw (0,\x) -- (30,\x);
 \node[rotate=90, attention] at (15,54) {{\small Davi}};
 \node[rotate=90, attention] at (15,42) {{\small Davi}};
 \node[rotate=90, attention] at (15,30) {{\small Edison}};
 \node[rotate=90, attention] at (15,18) {{\small Edison}};
 \node[rotate=90, attention] at (15,6) {{\small Edison}};
  \node[rotate=90, attention] at (15,6) {{\small\phantom{Amarildo}}};

\end{tikzpicture}
\end{center}


%Outra resposta possível: juntar as três tortas como na figura

% \begin{tikzpicture}[yscale=.45,xscale=.4, x=1mm,y=1mm]
%  \fill[cbyellow] (0,0) rectangle (60,30);
%   \fill[cborange] (60,0) rectangle (120,30);
%   \fill[cbbrown] (120,0) rectangle (180,30);
%   \foreach \x in {30,90,150}  \node [] at (\x,15) {\textbf{\small torta}};
% \end{tikzpicture}

% e dividir essa reunião em cinco partes iguais e distribuir uma dessas partes para cada amigo.

% \begin{tikzpicture}[yscale=.45,xscale=.4, x=1mm,y=1mm]
%   \fill[cbyellow] (0,0) rectangle (60,30);
%   \fill[cborange] (60,0) rectangle (120,30);
%   \fill[cbbrown] (120,0) rectangle (180,30);
%   \foreach \x in {36,72,108,144} \draw [dashed, very thick] (\x,-8) -- (\x,38); % quatro cortes.
%   %\foreach \x in {60,120} \draw[thin] (\x,0) -- (\x,30); % fronteira entre as tortas.
% \foreach \x/\n in {18/Amarildo, 54/Beto, 90/Carlos, 126/Davi, 162/Edison}  \node [ ] at (\x,15) {\textbf{\small \n}};
% \end{tikzpicture}
    \item[b)]
\begin{enumerate}[I)]
          \item Três quintos.
          \item Seis quintos (ou uma torta inteira e um quinto de torta).
          \item Nove quintos (ou uma torta inteira e quatro quintos de torta).
          \item Doze quintos (ou duas tortas inteiras e dois quintos de torta).
          \item Quinze quintos (ou três tortas inteiras).
\end{enumerate}

    \item[c)]     A quantidade de torta que cada amigo recebeu não pode ser menor do que um quinto de torta pois, se isto acontecesse, a quantidade total de torta recebida pelos cinco amigos seria menor do que cinco quintos de torta, isto é, seria menor do que uma torta inteira, o que não é o caso. Um argumento análogo mostra que a quantidade de torta que cada amigo recebeu não pode ser menor do que dois quintos de torta.
    \item[d)]       A quantidade de torta que cada amigo recebeu não pode ser maior do que três quintos de torta pois, se isto acontecesse, a quantidade total de torta recebida pelos cinco amigos seria maior do que quinze quintos de torta, isto é, seria maior do que três tortas inteiras, o que não é o caso. Um argumento análogo mostra que a quantidade de torta que cada amigo recebeu não pode ser maior do que quatro quintos de torta.
\end{enumerate} %d


\end{solucao}



\begin{objetivos}[label=chap2-ativ4]{}{}
\begin{itemize} %s
\item Compreender e usar a expressão ``$n$ oitavos de'' como forma de identificar a quantidade equivalente a $n$ cópias de um oitavo da unidade, incluindo os casos em que $n$ é maior do que oito.
\item Reconhecer que uma mesma quantidade pode ser expressa por frações equivalentes de uma mesma unidade (por exemplo, ``meia torta'' e ``quatro oitavos de torta'' representam a mesma quantidade de torta).
\end{itemize} %s
\end{objetivos}

\begin{orientacoes}{}{}

\begin{itemize} %s
    \item Recomenda-se que a atividade seja desenvolvida em grupos de 3 a 5 alunos.
    \item As diversas soluções apresentadas pelos diferentes grupos devem ser discutidas com a turma inteira.
    \item É importante que a discussão conduza os alunos ao uso de oitavos:  ``quatro oitavos'', ``dez oitavos'' e ``uma torta e dois oitavos''.
     \item As respostas esperadas para o Item c) podem surgir na resolução do Item b). Caso isso aconteça, recomenda-se que as frações corretas correspondentes a $4$ fatias de uma torta (metade de uma torta, dois quartos de uma torta, quatro oitavos de uma torta, etc.) sejam reconhecidas como tal, mas que, conforme solicitado pelo enunciado, a resposta deve ser dada em termos de oitavos.
     \item No Item c), é importante estimular o aluno a dar uma explicação para sua resposta: ``por que você pensou em metade de uma torta?'', ``Por que você pensou em dois quartos de uma torta?'' etc. Espera-se que os alunos expressem que estas frações expressam a mesma quantidade da unidade torta. Não se pretende usar a nomenclatura ``frações equivalentes''.
\end{itemize}

%  \noindent {\bf Classificações:}\vspace{.1cm}
% \begin{itemize} %s
%     \item       Heid et al.: Conceito: identificar, descrever
%     \item       Nicely, Jr.: Nível 1: reconhecer
%     \item       UERJ: Observar: identificar, reconhecer
% \end{itemize} %s



\end{orientacoes}

\begin{solucao}{}{}
\begin{enumerate} [\quad a)] %s
    \item       Cada fatia é um oitavo de torta, pois cada torta está dividida em oito partes iguais.
    \item       Havia para a sobremesa quatro oitavos de torta.
    \item       Meia torta, pois quatro fatias de torta têm a mesma quantidade de torta que meia torta.

  \begin{center}
  \includegraphics[width=175pt, keepaspectratio]{../figuras/licao02/ativ3_resposta.png}
  \end{center}

    \item       Algumas respostas possíveis: dez oitavos de torta;  uma torta inteira e dois oitavos de torta; uma torta inteira e um quarto de torta.
\end{enumerate} %s

\end{solucao}


\begin{objetivos}[label=chap2-ativ5]{}{}

  Compreender frações não unitárias (``m meios'', ``m terços'', etc.) em diferentes modelos visuais de frações como forma de identificar a quantidade equivalente a m cópias da fração $\frac{1}{n}$ (incluindo casos em que $m \geq n$).
\end{objetivos}

\begin{orientacoes}{}{}
  \begin{itemize} %s
    \item       Esta atividade pode ser resolvida individualmente, mas é essencial que seja discutida com toda a turma.
    \item       Observe que, enquanto que nas atividades anteriores cópias múltiplas da unidade já estavam naturalmente disponíveis (as duas barras de chocolate na Atividade 1, as três tortas salgadas na Atividade 2, as várias tortas divididas em oito partes na confeitaria da Atividade 3), nesta atividade, o aluno deve identificar frações a partir de uma única cópia da unidade, sem qualquer subdivisão registrada. Por exemplo, no item d), o aluno deve registrar nove meios de uma estrelinha, sem a subdivisão explicitada. Assim, a atividade oferece uma oportunidade para reforçar a compreensão de frações em um contexto diferente daquele em que a parte correspondente à fração é identificada e totalmente inserida em uma unidade, frequentemente já subdividida. Esse tipo de representação, muito associada ao significado parte/todo, pode limitar a compreensão de frações maiores que a unidade.
    \item       Nesta atividade, espera-se que o aluno identifique uma equipartição adequada da unidade que defina a fração unitária       $\frac{1}{m}$ da unidade para compor a parte colorida e que, então, tome a quantidade       $n$ correta desta fração unitária, mesmo no caso em que       $n > m$.
\end{itemize} %s

%  \noindent {\bf Classificações:}\vspace{.1cm}
% \begin{itemize} %s
%     \item       Heid et al.: Conceito: identificar
%     \item       Nicely, Jr.: Nível 1: reconhecer
%     \item       UERJ: Observar: identificar, nomear
% \end{itemize} %s

\end{orientacoes}

\begin{solucao}{}{}
  \begin{enumerate}[a)]
   \item Dois terços.
   \item Dois meios.
   \item Dois quintos.
   \item Nove meios.
   \item Oito sextos.
\end{enumerate}
  \end{solucao}


\end{multicols}

\subsection{Sobre o Organizando as Ideias}



Neste Organizando as Ideias, conceitos são  sistematizados mas também são introduzidas novas ideias. Espera-se que os alunos compreendam as frações não unitárias ($\frac{a}{b}$) como a identificação de uma quantidade igual à junção (por contagem ou por justaposição) de uma mesma fração unitária da unidade ($\frac{1}{b}$). 

Observe que esse entendimento é construído a partir de modelos contínuos e amparado por situações concretas. Assim, como explicado na introdução deste Organizando as Ideias, por exemplo, ``dois terços'' de uma unidade são obtidos pela reunião de duas partes correspondentes a ``um terço'' da mesma unidade.
Espera-se ainda que os alunos compreendam a notação simbólica matemática, tanto de fração unitária como de fração não unitária, bem como a nomenclatura numerador e denominador. 

A leitura das frações a partir do símbolo deve respeitar a proposta de abordagem. Assim, recomenda-se que $\frac{3}{5}$ seja lido como ``três quintos''. Nessa etapa de construção do conceito não cabe a leitura ``3 dividido por 5'', por ainda não termos como objetivo tal construto.
Também não deve ser feita a leitura ``abreviada'' e muito comum ``três sobre cinco''. Assim, de maneira geral, não se recomenda a leitura de $\frac{a}{b}$ como ``a sobre b'' nem como ``a dividido por b''.
Recomendamos que para a escrita $\frac{n}{d}$ utilizada ao final deste Organizando as Ideias, a leitura seja na forma ``fração de numerador $n$ e denominador $d$''.

Nesse contexto, surgiram frações que representam números naturais. Por exemplo, na Atividade 2, os alunos devem ter compreendido que a fração $\frac{3}{3}$ da barra de chocolate é uma barra inteira e a fração $\frac{6}{3}$ da barra de chocolate são duas barras. Não é recomendado, no entanto, que se utilize a nomenclatura ``fração aparente'', por considerarmos que o importante é o entendimento do seu significado e não do nome que damos a ela.

Também surgiram frações que representam quantidades maiores do que a unidade (evitando, pelo mesmo motivo acima, a nomenclatura ``frações impróprias'' ou ``frações mistas''). Por exemplo, $\frac{4}{3}$ de torta deve ser compreendido como ``uma torta e um terço de torta''. Reiteramos a recomendação de o professor evitar a escrita mais complexa $1 \frac{1}{3}$ de torta.

Observamos que as nomenclaturas ``fração imprópria'', ``fração aparente'' e ``fração mista'' - tradicionais nos livros didáticos brasileiros - aparecem para acomodar o pressuposto de que as únicas frações legítimas são aquelas em que o numerador é menor do que o denominador (um pressuposto que tem origem nas raízes históricas da construção do próprio conceito de fração).
A abordagem utilizada aqui, que estabelece frações gerais como a identificação de uma quantidade igual à junção de uma mesma fração unitária da unidade, rompe com essa tradição histórica ao dispensar esse pressuposto, com a grande vantagem de dar um tratamento geral e unificador às frações, sem a necessidade de nomear casos particulares.
Aliás, é essa abordagem que sobrevive: os termos ``fração imprópria'', ``fração aparente'' e ``fração mista'' nunca mais são usados no Ensino Médio nem no Ensino Superior.

Por fim, cabe ressaltar que a notação de fração pode não parecer natural para os alunos, porque é um símbolo composto por dois números de significados diferentes, além de estarem, um sobre o outro e separados por um traço. Isso contraria a escrita usual dos números naturais. Contudo, é importante lembrar que hoje essa é a notação mundialmente aceita, devendo, portanto, ser bem compreendida.

\Bg
\begin{multicols}{2}
  
\begin{objetivos}[label=chap2-ativ6]{}{}
  \begin{itemize} %d
\item Aplicar a simbologia matemática de frações.
\item Analisar criticamente diferentes formas de representação de frações: modelo visual, expressão verbal e símbolos matemáticos.
    %\item Comparar diversas maneiras de se representar uma fração (por extenso, simbolicamente e graficamente).
    %\item       Discutir aspectos dessas representações.
\end{itemize} %d

\end{objetivos}

\begin{orientacoes}
  \begin{itemize} %d
    \item       Essa é uma atividade que o aluno pode fazer individualmente.
    \item       É possível que os alunos utilizem frações equivalentes como resposta para um mesmo item. Por exemplo, as frações       $\frac{4}{12}$,       $\frac{2}{6}$ e       $\frac{1}{3}$ descrevem corretamente a quantidade de pizza consumida por Pedro. Nestes casos, dê oportunidade para que cada aluno explique como chegou à sua resposta pois, procedendo desta maneira, mesmo de forma pontual, os alunos perceberão que uma mesma quantidade pode ser descrita por frações com nomes diferentes, o que vai motivar o assunto ``frações equivalentes''     que será tratado na Lição 4.
    \item       Esta atividade procura mostrar uma das qualidades da notação simbólica matemática: expressar um conceito com economia de escrita. Ela permite encapsular detalhes, simplificar procedimentos, abstrair e generalizar conceitos. Assim, é muito importante fazer com que seus alunos se familiarizem com a notação simbólica matemática para frações: ela será fundamental nas lições sobre operações com frações, por exemplo.
      \item Antes de prosseguir para a próxima atividade é importante garantir que os estudantes estejam familiarizados com o uso dos símbolos matemáticos para representar frações.
\end{itemize} %d


 %  \vspace{.1cm}

%  \noindent {\bf Classificações:}\vspace{.1cm}
% \begin{itemize} %s
%     \item       Heid et al.: Produto: gerar
%     \item       Nicely, Jr.: Nível 5: converter (simbolizar)
%     \item       UERJ: Interpretar: discriminar
% \end{itemize} %s


%
%  \vspace*{\fill}
% \columnbreak
\end{orientacoes}

\begin{solucao}{}{}
\begin{center}
    \begin{tabular}{m{.15\textwidth}m{.15\textwidth}m{.15\textwidth}m{.15\textwidth}}

        \small Pedro & \small Isabella  &   \small Bernardo  &   \small Manuela  \\
      \hline
       \begin{tikzpicture}[x=1mm,y=1mm, scale=.7]
        \draw[fill=common, fill opacity=.3] (0,0) circle (10);
        \fill[attention] (90:10) arc (90:210:10) -- (0,0) -- cycle;
        \foreach \x in {0,30,...,150}\draw (\x:10) -- (\x:-10);
       \end{tikzpicture}&
       \begin{tikzpicture}[x=1mm,y=1mm, scale=.7]
        \draw[fill=common, fill opacity=.3] (0,0) circle (10);
        \fill[attention] (210:10) arc (210:360:10) -- (0,0) -- cycle;
        \foreach \x in {0,30,...,150}\draw (\x:10) -- (\x:-10);
       \end{tikzpicture}&
       \begin{tikzpicture}[x=1mm,y=1mm, scale=.7]
        \draw[fill=common, fill opacity=.3] (0,0) circle (10);
        \fill[attention] (0:10) arc (0:60:10) -- (0,0) -- cycle;
        \foreach \x in {0,30,...,150}\draw (\x:10) -- (\x:-10);
       \end{tikzpicture}&
       \begin{tikzpicture}[x=1mm,y=1mm, scale=.7]
        \draw[fill=common, fill opacity=.3] (0,0) circle (10);
        \fill[attention] (60:10) arc (60:90:10) -- (0,0) -- cycle;
        \foreach \x in {0,30,...,150}\draw (\x:10) -- (\x:-10);
       \end{tikzpicture}\\
      \hline
      \centering  {\small quatro doze avos}  & \centering  {\small cinco doze avos}  & \centering  {\small dois doze avos}  & {\centering  {\small \hspace{.08cm} um \newline doze avos}}   \\
      \hline
       \centering $\frac{4}{12}$& \centering  $\frac{5}{12}$& \centering  $\frac{2}{12}$ & \centering  $\frac{1}{12}$
    \end{tabular}
\end{center}
\begin{enumerate} [\quad a)] %s
    \item       A que usa a notação simbólica matemática.
    \item       As respostas podem variar de pessoa para pessoa. No entanto, a justificativa deve ser coerente com a resposta. Discuta com a turma as diferentes respostas.
\end{enumerate} %s

\end{solucao}

%\newpage
%\begin{multicols}{2}

\begin{objetivos}[label=chap2-ativ7]{}{}
  \begin{itemize} %s
	\item Expressar as frações $\frac{1}{4}$ e $\frac{1}{8}$ em símbolos matemáticos.
	\item Comparar as frações um quarto e um oitavo a partir de modelos visuais.
\end{itemize} %s
\end{objetivos}

\begin{orientacoes}

\begin{itemize} %s
    \item Esta atividade pode ser resolvida individualmente, mas é essencial que seja discutida com toda a turma.
    \item Os estudantes compararam as frações um quarto e um oitavo em modelos retangulares na Atividade 8 da Lição 1, isso justifica a opção por um modelo circular nesta atividade.
    \item Em particular, incentive os alunos a argumentar, justificando a sua resposta.
    \item Conduza a discussão de modo a conseguirem reconhecer a relação inversa entre denominador (número de partes) e o tamanho de cada parte: quanto maior o denominador, maior é o número de partes em que foi repartida a pizza, logo menor o tamanho da parte.
      \item Embora não seja o objetivo da atividade, algum estudante pode reconhecer que uma fatia da primeira pizza tem o dobro da quantidade de uma fatia da segunda  pizza, ou seja, são necessárias duas fatias da segunda pizza para ter-se a mesma quantidade de pizza que uma fatia da primeira pizza.
\end{itemize}

%  \noindent {\bf Classificações:}\vspace{.1cm}:
% \begin{itemize} %s
%     \item       Heid et al.: Conceito: gerar
%     \item       Nicely, Jr.: Nível 3: comparar
%     \item       UERJ: Observar: Observar: identificar e reconhecer; Ordenar
% \end{itemize} %s


\end{orientacoes}

\begin{solucao}{}{}
\begin{enumerate} [\quad a)] %s
    \item $\frac{1}{4}$.
    \item $\frac{1}{8}$.
    \item Uma fatia da primeira pizza é maior do que uma fatia da segunda pizza: como partimos a segunda pizza em mais partes, cada fatia tem menos pizza.

\end{enumerate} %s
\begin{center}
       \begin{tikzpicture}[x=1mm,y=1mm, scale=.7]
        \draw[fill=common, fill opacity=.3] (0,0) circle (10);
        \fill[attention] (180:10) arc (180:270:10) -- (0,0) -- cycle;
        \foreach \x in {0,90}\draw (\x:10) -- (\x:-10);
       \end{tikzpicture} \quad \quad
       \begin{tikzpicture}[x=1mm,y=1mm, scale=.7]
        \draw[fill=common, fill opacity=.3] (0,0) circle (10);
        \fill[attention] (180:10) arc (180:270:10) -- (0,0) -- cycle;
        \foreach \x in {0,45,90, 135}\draw (\x:10) -- (\x:-10);
       \end{tikzpicture}
\end{center}
\end{solucao}


  \begin{objetivos}[label=chap2-ativ8]{}{}
  \begin{itemize} %s
    \item       Comparar frações com relação a uma fração de referência (no caso, a fração       $\frac{1}{2}$) usando modelos contínuos (área).
\end{itemize} %s


\end{objetivos}

\begin{orientacoes}
  \begin{itemize} %s
    \item       Essa é uma atividade que o aluno pode fazer individualmente.
    \item       Incentive seus alunos a darem justificativas para suas respostas, mesmo que informais.
\end{itemize} %s

%  \noindent {\bf Classificações:}\vspace{.1cm}
% \begin{itemize} %s
%     \item       Heid et al.: Conceito: identificar
%     \item       Nicely, Jr.: Nível 3: comparar
%     \item       UERJ: Observar: identificar e reconhecer; Ordenar
% \end{itemize} %s
\end{orientacoes}

\begin{solucao}{}{}
\begin{enumerate} [\quad a)] %s
    \item A parte pintada é igual a $\frac{5}{10}$ da figura, ou seja, metade da figura.
    \item A parte pintada é igual a $\frac{4}{10}$, que é menor do que       $\frac{5}{10}$ da figura, metade da figura.
    \item  A parte pintada é igual a $\frac{6}{10}$ e é maior do que       $\frac{5}{10}$ da figura, que é a metade da figura.
\end{enumerate} %s

\end{solucao}




\begin{objetivos}[label=chap2-ativ9]{}{}
  \begin{itemize} %s
    \item       Representar frações não unitárias descritas com símbolos matemáticos em diversos modelos de área, incluindo casos em que as subdivisões apresentadas não coincidem com o denominador da fração dada.
    \item       Identificar a fração complementar de uma fração da unidade usando símbolos matemáticos.
    \item       Reconhecer (e gerar) oitavos como metades de quartos, sextos como metades de terços e décimos como metades de quintos, preparando-se assim para a discussão sobre equivalência de frações que será feita na Lição 4.
\end{itemize} %s
\end{objetivos}

\begin{orientacoes}
\begin{itemize} %s
    \item       Essa é uma atividade que o aluno pode fazer individualmente.
    \item       Observe que os três últimos itens constituem uma extensão natural da Atividade 8 da Lição 1. Isto é, aplicam o fato de que, para uma mesma unidade, sexto é metade de terço, oitavo é metade de quarto e décimo é metade de quinto.
    \item       Não se espera, nem se recomenda, que, para os três últimos itens desta atividade, os alunos usem alguma medida para fazer, de forma precisa, a partição de quartos e quintos em oitavos e décimos, respectivamente. O objetivo é que façam a partição livremente e de forma coerente.
    \item Os alunos podem pintar as partes de formas diferentes: estas, por exemplo, não precisam ser justapostas.
    \item Procure apresentar e discutir com a turma mais do que uma solução para cada item, reforçando assim a ideia proposta na Atividade 9 da Lição 1.
\end{itemize} %s


%  \noindent {\bf Classificações:}\vspace{.1cm}
% \begin{itemize} %s
%     \item       Heid et al.: Conceito: elaborar/identificar; Produto: gerar
%     \item       Nicely, Jr.: Nível 5: converter (simbolizar), gerar
%     \item       UERJ: Interpretar: discriminar, compor e decompor
% \end{itemize} %s

%\newpage

  %\begin{multicols}{2}
\end{orientacoes}
\end{multicols}

\begin{solucao}{}{}

\begin{center}
    \begin{tabular}{|m{0.15\textwidth}|m{0.3\textwidth}|m{0.25\textwidth}|}
\hline
      \centering A pintar  & \centering figura &\quad \quad sem ser pintada  \\
      \hline
 \centering $\dfrac{5}{6}$& \centering
                                    \begin{tikzpicture}[x=1mm,y=1mm]
                                    \foreach \x in {120,180,...,360} \fill[attention] (\x:8)--(\x+60:8)--(0,0)--cycle;
                                    \fill[common, opacity=.3] (60:8) -- (120:8) -- (0,0) -- cycle;
                                    \foreach \x in {0,60,...,300}{ \draw (0,0)--(\x:8);\draw (\x:8)--(\x+60:8);}
                                   \end{tikzpicture}
&  $$\dfrac{1}{6}$$ \\
    \hline
     \centering $\dfrac{3}{4}$&  \centering \begin{tikzpicture}[x=1mm,y=1mm]
                                    \draw[fill=common, fill opacity=.3] (0,0) circle (8);
                                    \fill[attention] (0:8) arc (0:270:8) -- (0,0) -- cycle;
                                    \draw (0:8)--(180:8);
                                    \draw (90:8)--(270:8);
                                   \end{tikzpicture}
                                   & $$\dfrac{1}{4}$$\\
    \hline
     \centering $\dfrac{2}{5}$&   \centering
                                    \begin{tikzpicture}[x=1mm,y=1mm,scale=.8]
                                    \draw[fill=common, fill opacity=.3] (0,0) rectangle (25,16);
                                    \fill[attention] (0,0) rectangle (10,16);
                                    \foreach \x in {5,10,15,20}{\draw (\x,0)--(\x,16);}
                                  \end{tikzpicture} ou
                                \begin{tikzpicture}[x=1mm,y=1mm,scale=.8]
                                    \draw[fill=common, fill opacity=.3] (0,0) rectangle (25,16);
                                    \fill[attention] (0,0) rectangle (5,16);
                                    \fill[attention] (10,0) rectangle (15,16);
                                    \foreach \x in {5,10,15,20}{\draw (\x,0)--(\x,16);}
                                   \end{tikzpicture}
                                   & $$\dfrac{3}{5}$$ \\
    \hline
     \centering $\dfrac{2}{3}$&  \centering \begin{tikzpicture}[x=1mm,y=1mm]
                                    \foreach \x in {120,180,...,300} \fill[attention] (\x:8)--(\x+60:8)--(0,0)--cycle;
                                    \fill[common, opacity=.3] (60:8) -- (120:8) -- (0,0) -- (0:8)-- cycle;
                                    \foreach \x in {0,60,...,300}{ \draw (0,0)--(\x:8);\draw (\x:8)--(\x+60:8);}
                                  \end{tikzpicture} ou
                                \begin{tikzpicture}[x=1mm,y=1mm]
                                  \foreach \x in {120,180,240} \fill[attention] (\x:8)--(\x+60:8)--(0,0)--cycle;
                                   \fill[attention] (0:8)--(60:8)--(0,0)--cycle;
                                   \fill[common, opacity=.3] (60:8) -- (120:8) -- (0,0)-- cycle;
                                   \fill[common, opacity=.3] (300:8) -- (360:8) -- (0,0)-- cycle;
                                    \foreach \x in {0,60,...,300}{ \draw (0,0)--(\x:8);\draw (\x:8)--(\x+60:8);}
                                   \end{tikzpicture}
                                   & $$\dfrac{1}{3}$$ \\
    \hline
     \centering $\dfrac{3}{8}$&   \centering \begin{tikzpicture}[x=1mm,y=1mm]
                                    \draw[fill=common, fill opacity=.3] (0,0) circle (8);
                                    \fill[attention] (0:8) arc (0:135:8) -- (0,0) -- cycle;
                                    \foreach \x in {0,45,90,135} \draw (\x:8)--(\x:-8);
                                  \end{tikzpicture} ou
                                  \begin{tikzpicture}[x=1mm,y=1mm]
                                    \draw[fill=common, fill opacity=.3] (0,0) circle (8);
                                    \fill[attention] (180:8) arc (180:225:8) -- (0,0) -- cycle;
                                    \fill[attention] (0:8) arc (0:45:8) -- (0,0) -- cycle;
                                    \fill[attention] (270:8) arc (270:315:8) -- (0,0) -- cycle;
                                    \foreach \x in {0,45,90,135} \draw (\x:8)--(\x:-8);
                                  \end{tikzpicture} & $$\dfrac{5}{8}$$ \\
    \hline
     \centering $\dfrac{9}{10}$& \centering \begin{tikzpicture}[x=1mm,y=1mm,scale=.8]
                                    \draw[fill=common, fill opacity=.3] (20,8) rectangle (25,16);
                                    \filldraw[fill=attention] (0,0) rectangle (20,16);
                                    \filldraw[fill=attention] (20,0) rectangle (25,8);
                                    \foreach \x in {5,10,15,20}{\draw (\x,0)--(\x,16);}
                                    \draw (0,8) -- (25,8);
                                   \end{tikzpicture}
                                   & $$\dfrac{1}{10}$$ \\
    \hline
    \end{tabular}
  \end{center}
\end{solucao}

\begin{multicols}{2}
\begin{objetivos}[label=chap2-ativ10]{}{}
\begin{itemize} %s
  %  \item       Representar com notação simbólica matemática frações não unitárias em modelos tridimensionais no contexto de volume.
    \item       Analisar e resolver um problema no contexto da reunião de partes correspondentes a frações unitárias com mesmo denominador.
\end{itemize} %s
\end{objetivos}

\begin{orientacoes}
    \begin{itemize} %s
    \item Essa é uma atividade que o aluno pode fazer individualmente.
    \item As diversas soluções apresentadas devem ser discutidas com a turma inteira.
    \item Caso os estudantes se refiram a cada uma das oito partes em que foi dividido cada um dos três copos de maneira diferente de ``oitavos'', pergunte a eles ``\textit{a que fração do copo essa parte corresponde?}'' e então peça que se esforcem para usar ``oitavos'' ao invés da expressão que estavam usando.
    \item Avalie a necessidade de apresentar os seguintes problemas preliminares:
      \begin{enumerate}[a)]
      \item Indique a fração da capacidade do copo que está com água em cada um dos três copos.
      \item Qual é a fração da capacidade do copo que corresponde a toda água que está nos três copos?
      \end{enumerate}
    \item É possível que os alunos utilizem frações equivalentes como resposta para um mesmo item. Por exemplo, para o copo (3), as frações $\frac{4}{8}$, $\frac{2}{4}$ e $\frac{1}{2}$ são respostas corretas. Nesses casos, dê a oportunidade para que cada aluno explique como chegou à sua resposta. Procedendo desta maneira, mesmo que de forma pontual, os alunos perceberão que uma mesma quantidade pode ser descrita por frações com numeradores e denominadores diferentes, um preparo para o assunto ``frações equivalentes'' que será tratado na Lição 4.
\end{itemize} %s

%  \noindent {\bf Classificações:}\vspace{.1cm}
% \begin{itemize} %s
%     \item       Heid et al.: Conceito: elaborar/identificar; Produto: gerar; Raciocínio: justificar
%     \item       Nicely, Jr.: Nível 6: analisar/justificar
%     \item       UERJ: Interpretar: discriminar, compor e decompor; Analisar: transferir conhecimentos
% \end{itemize} %s

\end{orientacoes}

\begin{solucao}{}{}
  Não é possível armazenar a água dos três copos em um único copo sem que o mesmo transborde, pois a água do primeiro copo ocupa 3 oitavos de sua capacidade, a água do segundo copo ocupa 2 oitavos de sua capacidade e a água do terceiro copo ocupa 4 oitavos de sua capacidade, a água dos três copos, juntos, ocupa       $3 + 2 + 4 = 9$ oitavos da capacidade do copo e qualquer copo só consegue armazenar no máximo $8$ oitavos de sua capacidade.

  Caso tenha decidido aceitar a sugestão das perguntas preliminares, as respostas são:
\begin{enumerate} [\quad a)] %s
    \item       (1):       $\dfrac{3}{8}$. (2):       $\dfrac{2}{8}$. (3):       $\dfrac{4}{8}$.
    \item             $\dfrac{9}{8}$.
\end{enumerate} %s

\end{solucao}

%\newpage
%\begin{multicols}{2}

\begin{objetivos}[label=chap2-ativ11]{}{}
  \begin{itemize} %d
    \item       Recompor a unidade a partir de uma fração dada em modelo contínuo e em linguagem simbólica, incluindo o caso de frações maiores que a unidade.
    \item       Relacionar a fração correspondente à parte apresentada à quantidade necessária dessas partes para compor a unidade. Assim, por exemplo, para compor a unidade a partir de       $\frac{2}{3}$ da unidade, basta repartir esta fração em 2 partes iguais (para recuperar a fração unitária       $\frac{1}{3}$) e, então, justapor 3 cópias de uma destas partes.
\end{itemize} %d
\end{objetivos}

\begin{orientacoes}
  \begin{itemize} %d
    \item Recomenda-se que a atividade seja discutida em grupos de 3 a 5 alunos e respondida individualmente.
    %\item A exemplo das Atividades 5 e 7 da Lição 1, é importante ter em mente que existem várias soluções para cada item.
 \item A exemplo das Atividades 5 e 7 da Lição 1, é importante ter em mente que a forma da unidade pode variar pois a unidade pode ser representada de várias maneiras.
    \item Estimule os alunos a reconhecer (e a fazer) mais do que uma representação para a unidade em cada item.
    \item Caso seja necessário fazer alguma partição, não se espera nem se recomenda que os alunos usem alguma medida. Não se espera precisão, uma partição coerente será suficiente. 
\end{itemize} %d
\end{orientacoes}
%
%\newpage

%  \noindent {\bf Classificações:}\vspace{.1cm}
% \begin{itemize} %s
%     \item       Heid et al.: Produto: gerar
%     \item       Nicely, Jr.: Nível 5: relacionar
%     \item       UERJ: Interpretar: compor e decompor
% \end{itemize} %s
%\columnbreak
%\vfill

\begin{solucao}{}{}
\begin{center}
  \begin{tabular}{|m{0.2\textwidth}|m{0.3\textwidth}|c|}
    \hline
     \centering Fração da unidade  & \centering  Figura correspondente à fração da unidade  & Unidade \\
    \hline \hline
     \centering $\dfrac{1}{2}$ & \centering \begin{tikzpicture}[x=1mm,y=1mm]
                                    \draw[fill=common, fill opacity=.3] (0,0) rectangle (12,6);
                                   \end{tikzpicture}
				&\parbox[c][1.1cm][c]{24mm}{\begin{tikzpicture}[x=1mm,y=1mm]
 \draw[fill=common, fill opacity=.3,  ] (0,0) rectangle (24,6);
 \end{tikzpicture}} \\
    \hline
     \centering $\dfrac{4}{2}$ &   \centering \begin{tikzpicture}[x=1mm,y=1mm]
                                    \draw[fill=common, fill opacity=.3,  ] (0,0) rectangle (12,6);
                                   \end{tikzpicture}
                                   & \parbox[c][1.1cm][c]{24mm}{\centering \begin{tikzpicture}[x=1mm,y=1mm]
                                    \draw[fill=common, fill opacity=.3] (0,0) rectangle (6,6);
                                   \end{tikzpicture}} \\
    \hline
     \centering $\dfrac{3}{2}$   &  \centering \begin{tikzpicture}[x=1mm,y=1mm]
                                    \draw[fill=common, fill opacity=.3,  ] (0,0) rectangle (12,6);
                                   \end{tikzpicture}
                                   & \parbox[c][1.1cm][c]{24mm}{\centering \begin{tikzpicture}[x=1mm,y=1mm]
                                    \draw[fill=common, fill opacity=.3] (0,0) rectangle (8,6);
                                   \end{tikzpicture}}  \\
    \hline
     \centering $\dfrac{2}{3}$ &  \centering \begin{tikzpicture}[x=1mm,y=1mm]
                                    \draw[fill=common, fill opacity=.3,  ] (0,0) rectangle (12,6);
                                  \end{tikzpicture}
                                  & \parbox[c][1.1cm][c]{24mm}{\centering \begin{tikzpicture}[x=1mm,y=1mm]
\draw[fill=common, fill opacity=.3] (0,0) rectangle (18,6);
\end{tikzpicture}} \\
\hline
\centering $\dfrac{1}{2}$ &  \centering \begin{tikzpicture}[x=1mm,y=1mm]
\draw[fill=common, fill opacity=.3,  ] (0,0) arc (0:180:6) -- cycle;
\end{tikzpicture}
& \parbox[c][1.3cm][c]{24mm}{\centering \begin{tikzpicture}[x=1mm,y=1mm]
\draw[fill=common, fill opacity=.3] (0,0) circle (6);
\end{tikzpicture}}
\\
\hline
\centering $\dfrac{4}{2}$ &  \centering \begin{tikzpicture}[x=1mm,y=1mm]
\draw[fill=common, fill opacity=.3,  ] (0,0) arc (0:180:6) -- cycle;
\end{tikzpicture}
& \parbox[c][1.1cm][c]{24mm}{\centering \begin{tikzpicture}[x=1mm,y=1mm]
\draw[fill=common, fill opacity=.3] (-6,6) arc (90:180:6) -- (-6,0)-- cycle;
\end{tikzpicture}}
\\
\hline
\centering $\dfrac{3}{2}$ &  \centering \begin{tikzpicture}[x=1mm,y=1mm]
\draw[fill=common, fill opacity=.3,  ] (0,0) arc (0:180:6) -- cycle;
\end{tikzpicture}
&
\parbox[c][1.1cm][c]{24mm}{\centering \begin{tikzpicture}[x=1mm,y=1mm]
\draw[fill=common, fill opacity=.3,  ] (0,0) -- (60:6) arc (60:180:6) -- cycle;
\end{tikzpicture}}
\\
\hline
\centering $\dfrac{2}{3}$ &  \centering \begin{tikzpicture}[x=1mm,y=1mm]
\draw[fill=common, fill opacity=.3,  ] (0,0) arc (0:180:6) -- cycle;
                                   \end{tikzpicture} & \parbox[c][1.3cm][c]{24mm}{\centering \begin{tikzpicture}[x=1mm,y=1mm]
\draw[fill=common, fill opacity=.3] (0:6) arc (0:270:6) -- (0,0) -- cycle;
\end{tikzpicture}} \\
    \hline
\centering $\dfrac{1}{2}$ & \centering \begin{tikzpicture}[x=1mm,y=1mm]
 \draw[fill=common, fill opacity=.3,  ] (0,0) rectangle (12,1);
                                  \end{tikzpicture}
& \parbox[c][1.1cm][c]{24mm}{\centering \begin{tikzpicture}[x=1mm,y=1mm]
                                    \draw[fill=common, fill opacity=.3] (0,0) rectangle (24,1);
                                   \end{tikzpicture}} \\
    \hline
    \centering $\dfrac{4}{2}$ &  \centering \begin{tikzpicture}[x=1mm,y=1mm]
 \draw[fill=common, fill opacity=.3,  ] (0,0) rectangle (12,1);
                                  \end{tikzpicture}
& \parbox[c][1.1cm][c]{24mm}{\centering \begin{tikzpicture}[x=1mm,y=1mm]
                                    \draw[fill=common, fill opacity=.3] (0,0) rectangle (6,1);
                                   \end{tikzpicture}} \\
    \hline
      \centering $\dfrac{3}{2}$ &  \centering \begin{tikzpicture}[x=1mm,y=1mm]
\draw[fill=common, fill opacity=.3,  ] (0,0) rectangle (12,1);                 \end{tikzpicture}                                                                                            & \parbox[c][1.1cm][c]{24mm}{\centering \begin{tikzpicture}[x=1mm,y=1mm]
                                    \draw[fill=common, fill opacity=.3] (0,0) rectangle (8,1);
                                   \end{tikzpicture}} \\
    \hline
\centering $\dfrac{2}{3}$ &  \centering \begin{tikzpicture}[x=1mm,y=1mm]
\draw[fill=common, fill opacity=.3,  ] (0,0) rectangle (12,1);
\end{tikzpicture} & \parbox[c][1.1cm][c]{24mm}{\centering \begin{tikzpicture}[x=1mm,y=1mm]
                                    \draw[fill=common, fill opacity=.3,  ] (0,0) rectangle (18,1);
                                   \end{tikzpicture}}  \\
    \hline
        \centering $\dfrac{1}{2}$ &  \centering \begin{tikzpicture}[x=1mm,y=1mm]
                                      \draw[fill=common, fill opacity=.3,  ] (0:4) -- (60:4)--(120:4)-- (180:4)--(240:4)--(300:4)--cycle;
                                    \end{tikzpicture}                                                                                            & \parbox[c][1.1cm][c]{24mm}{\centering \begin{tikzpicture}[x=1mm,y=1mm]
%\draw[fill=common, fill opacity=.3] (0:4) -- (60:4) -- (120:4) -- (180:4) -- (240:4) -- (300:4) -- cycle;
%\draw[fill=common, fill opacity=.3, shift={(-6,{2*sqrt(3)})}] (180:4) -- (0:4) -- (300:4) -- (240:4)--cycle;
%\draw[fill=common, fill opacity=.3, shift={(-6,{-2*sqrt(3)})}] (180:4) -- (0:4) -- (60:4) -- (120:4)--cycle;
                                    \draw[fill=common, fill opacity=.3] (-10,-3.42) -- (240:4) -- (300:4) -- (0:4) -- (60:4) -- (120:4) --+ (-8,0) -- (-8,0) -- cycle;
                                    \draw [dashed, gray] (180:8) -- (180:4) -- (240:4);
\draw [dashed, gray] (180:4) -- (120:4);
                                  \end{tikzpicture}}\\

%                                      ou \begin{tikzpicture}[x=1mm,y=1mm]
%                                     \draw[fill=common, fill opacity=.3] (0:4) -- (60:4)--(120:4)-- (180:4)--(240:4)--(300:4)--cycle;
%                                     \draw[fill=common, fill opacity=.3, shift={(8,0)} ] (0:4) -- (60:4)--(120:4)-- (180:4)--(240:4)--(300:4)--cycle;
%                                     \end{tikzpicture}
     \hline
     \centering $\dfrac{4}{2}$ &  \centering \begin{tikzpicture}[x=1mm,y=1mm]
                                    \draw[fill=common, fill opacity=.3,  ] (0:4) -- (60:4)--(120:4)-- (180:4)--(240:4)--(300:4)--cycle;
                                   \end{tikzpicture} & \parbox[c][1.1cm][c]{24mm}{\centering \begin{tikzpicture}[x=1mm,y=1mm]
                                   \draw[fill=common, fill opacity=.3] (180:4) -- (0:4) -- (60:4) -- (120:4)--cycle;
\end{tikzpicture}}  \\
\hline
\centering $\dfrac{3}{2}$ &  \centering  \begin{tikzpicture}[x=1mm,y=1mm]
\draw[fill=common, fill opacity=.3,  ] (0:4) -- (60:4)--(120:4)-- (180:4)--(240:4)--(300:4)--cycle;
\end{tikzpicture}
& \parbox[c][1.1cm][c]{24mm}{\centering \begin{tikzpicture}[x=1mm,y=1mm]
\draw[fill=common, fill opacity=.3] (0,0) -- (0:4) -- (60:4) -- (120:4)-- (180:4) -- (240:4) -- cycle;
%\draw[fill=common, fill opacity=.3] (240:4)  -- (300:4)-- (0:4)-- (0,0) --cycle;
%\draw[ ] (0:4) -- (60:4)--(120:4)-- (180:4)--(240:4)--(300:4)--cycle;
%\draw (0,0) -- (120:4);
\end{tikzpicture}} \\
    \hline
      \centering $\dfrac{2}{3}$ &  \centering \begin{tikzpicture}[x=1mm,y=1mm]
                                    \draw[fill=common, fill opacity=.3,  ] (0:4) -- (60:4)--(120:4)-- (180:4)--(240:4)--(300:4)--cycle; \end{tikzpicture}                                         &  \parbox[c][1.1cm][c]{24mm}{\centering \begin{tikzpicture}[x=1mm,y=1mm]
%\draw[fill=common, fill opacity=.3] (0:4) -- (60:4)--(120:4)-- (180:4) -- (240:4) -- (300:4) -- cycle;
%\draw[fill=common, fill opacity=.3, shift={(-6,{-2*sqrt(3)})}] (180:4) -- (0:4) -- (60:4) -- (120:4)--cycle;
%\draw[ ] (0:4) -- (60:4)--(120:4)-- (180:4)--(240:4)--(300:4)--cycle;
     \end{tikzpicture}}  \\
    \hline
  \end{tabular}
\end{center}
\end{solucao}
\newpage

\begin{objetivos}[label=chap2-ativ12]{}{}
\begin{itemize} %s
    \item       Marcar em uma semirreta pontos cujas distâncias até um ponto de referência são frações do comprimento de um segmento dado.
\end{itemize} %s
\end{objetivos}

\begin{orientacoes}
  \begin{itemize} %s
    \item       Esta é uma atividade que pode ser realizada individualmente.
    \item       Esta é uma atividade preparatória para a representação de frações na reta numérica, assunto da próxima lição.
    \item       Observe que, nesta atividade, as distâncias estão associadas aos segmentos determinados pelos percursos dos carrinhos na pista, e correspondem a frações da distância percorrida pelo carrinho de Lucas, que assume papel de unidade.
      % não se espera precisão, bastam marcações coerentes.
    \item       Não se espera, nem se recomenda, que as marcações feitas pelos alunos na pista sejam feitas com precisão, marcações coerentes são suficientes. Contudo, é preciso ficar atento para que as marcações dos carrinhos de Heitor e de Lorenzo coincidam. A mesma observação se aplica aos carrinhos de Rafael, Samuel e de Guilherme. De fato, três metades do percurso de Lucas coincide com seis quartos do percurso de Lucas.
    \item       Aqui, a definição de fração não unitária como justaposição de frações unitárias pode ser usada para justificar o porquê, por exemplo, de os carrinhos de Rafael e de Samuel terem parado na mesma posição.
    \item       Assim, espera-se que a distância percorrida pelo carrinho de Matheus (item a) seja associada à metade do segmento que identifica a distância percorrida pelo carrinho de Lucas, que corresponde à unidade e está destacado em vermelho na imagem. Já a distância percorrida pelo carrinho de Heitor (item b) deve ser associada à justaposição de       $3$ segmentos correspondentes à distância percorrida pelo carrinho de Matheus. Espera-se que as demais distâncias sejam obtidas de forma semelhante. Cabe destacar, no entanto, que para determinar as distâncias percorridas pelos carrinhos de Lorenzo e de Samuel, será necessário determinar       $\frac{1}{4}$ e       $\frac{1}{3}$ da unidade, respectivamente.
  %  \item       De forma geral, se $d$ é a distância percorrida pelo carrinho de Lucas, então a partição em $2$ partes iguais de um segmento $u$ cujo comprimento é $d$ determina dois segmentos congruentes $s$ e $s'$ que correspondem à $\frac{1}{2}$ de       $u$ e cujos comprimentos são, portanto, iguais a       $\frac{1}{2}$ de       $d$. A justaposição de       $2$ cópias de       $s$ ($\frac{2}{2}$ de       $u$) tem comprimento       $d$ e, sendo assim, a justaposição de       $4$ cópias de       $s$ ($\frac{4}{2}$ de $u$) tem comprimento       $2d$. Do mesmo modo, se       $t$ é um segmento que corresponde à       $\frac{1}{3}$ de       $u$, então a justaposição de       $3$ cópias de       $t$ ($\frac{3}{3}$ de       $u$) tem comprimento       $d$ e, em consequência, a justaposição de       $6$ cópias de $t$ ($\frac{6}{3}$ de $u$) tem comprimento $2d$. Assim, os carrinhos de Rafael e de Samuel percorreram a mesma distância ($2d$) e, como eles saíram do mesmo ponto de largada, suas posições finais são iguais.
\end{itemize} %s


%  \noindent {\bf Classificações:}\vspace{.1cm}
% \begin{itemize} %s
%     \item       Heid et al.: Produto: gerar
%     \item       Nicely, Jr.: Nível 5: relacionar
%     \item       UERJ: Interpretar: compor e decompor
% \end{itemize} %s


%   Para a pergunta sobre as posições dos carrinhos de Rafael e Samuel:
% \begin{itemize} %s
%     \item       Heid et al.: Racioncínio: justificar
%     \item       Nicely, Jr.: Nível 6: explicar
%     \item       UERJ: Interpretar: explicar, compor e decompor
% \end{itemize} %s
\end{orientacoes}

\begin{solucao}{}{}

Do item a) sabe-se que o carrinho de Matheus só conseguiu ir até a metade da distância percorrida pelo carrinho de Lucas. 

\hspace{-15mm} 

\begin{center}
\begin{tikzpicture}[every node/.style={scale=.7}]

\tikzstyle{circ}=[draw, circle, inner sep=.075cm, node distance=1.75cm, thick];

\tikzstyle{circ2}=[circle, inner sep=.075cm, node distance=1.75cm, thick];

\node (larg) [circ, label=below:Largada] {} ;

\node (1) [circ, right of=larg, label=below:Matheus] {} node [above of=1, node distance=.5cm]{\includegraphics[width=1cm]{../figuras/licao02/carrinho2.png}};

\node (2) [circ, right of=1, label=below:Lucas] {} node [above of=2, node distance=.5cm]{\includegraphics[width=1cm]{../figuras/licao02/carrinho2.png}};

\node (3) [right of=2, circ2] {} ;

\node (4) [right of=3, circ2] {};

\node (5) [right of=4, circ2] {};

\node (6) [right of=5, circ2] {};

\path[thick]

(larg) edge (1)
(1) edge (2)
(2) edge (6)
;
\end{tikzpicture}
\end{center}

  Do item b), o carrinho de Heitor conseguiu ir até $\frac{3}{2}$ da ``distância percorrida pelo carrinho de Lucas'' (a unidade).  Isso permite marcar a posição do carrinho de Heitor justapondo-se 3 metades da unidades a partir do ponto de partida.

  \hspace{-15mm} 

\begin{center}
\begin{tikzpicture}[every node/.style={scale=.7}]

\tikzstyle{circ}=[draw, circle, inner sep=.075cm, node distance=1.75cm, thick];

\tikzstyle{circ2}=[circle, inner sep=.075cm, node distance=1.75cm, thick];

\node (larg) [circ, label=below:Largada] {} ;

\node (1) [circ, right of=larg, label=below:Matheus] {} node [above of=1, node distance=.5cm]{\includegraphics[width=1cm]{../figuras/licao02/carrinho2.png}};

\node (2) [circ, right of=1, label=below:Lucas] {} node [above of=2, node distance=.5cm]{\includegraphics[width=1cm]{../figuras/licao02/carrinho2.png}};

\node (3) [circ, right of=2, label=below:Heitor] {} node [above of=3, node distance=.5cm] {\includegraphics[width=1cm]{../figuras/licao02/carrinho2.png}};

\node (4) [circ2, right of=3] {};

\node (5) [circ2, right of=4] {};

\node (6) [circ2, right of=5] {};

% \node (7) [circ, right of=6, label=below:Guilherme] {} node [above of=7, node distance=.5cm] {\includegraphics[width=1cm]{../figuras/licao02/carrinho2.png}};

\path[thick]

(larg) edge (1)
(1) edge (2)
(2) edge (3)
(3) edge (6)
% (6) edge (7)
;
\end{tikzpicture}
\end{center}


  Os itens c), d) e e) afirmam que os carrinhos de Rafael, Enzo e Nicolas conseguiram ir até $\frac{4}{2}$, $\frac{5}{2}$ e $\frac{6}{2}$  da distância percorrida pelo carrinho de Lucas, respectivamente. Então pode-se marcar as distâncias percorridas pelos carrinhos de Rafael, Enzo e Nicolas justapondo-se, respectivamente, 4, 5 e 6 meios da unidade.

  \hspace{-15mm} 

\begin{center}
\begin{tikzpicture}[every node/.style={scale=.7}]

\tikzstyle{circ}=[draw, circle, inner sep=.075cm, node distance=1.75cm, thick];

\tikzstyle{circ2}=[circle, inner sep=.075cm, node distance=1.75cm, thick];

\node (larg) [circ, label=below:Largada] {} ;

\node (1) [circ, right of=larg, label=below:Matheus] {} node [above of=1, node distance=.5cm]{\includegraphics[width=1cm]{../figuras/licao02/carrinho2.png}};

\node (2) [circ, right of=1, label=below:Lucas] {} node [above of=2, node distance=.5cm]{\includegraphics[width=1cm]{../figuras/licao02/carrinho2.png}};

\node (3) [circ, right of=2, label=below:Heitor] {} node [above of=3, node distance=.5cm] {\includegraphics[width=1cm]{../figuras/licao02/carrinho2.png}};

\node (4) [circ, right of=3, label=below:Rafael] {} node [above of=4, node distance=.5cm] {\includegraphics[width=1cm]{../figuras/licao02/carrinho2.png}};

\node (5) [circ, right of=4, label=below:{Enzo}] {} node [above of=5, node distance=.5cm] {\includegraphics[width=1cm]{../figuras/licao02/carrinho2.png}};

\node (6) [circ, right of=5, label=below:{Nicolas}] {} node [above of=6, node distance=.5cm] {\includegraphics[width=1cm]{../figuras/licao02/carrinho2.png}};

% \node (7) [circ, right of=6, label=below:Guilherme] {} node [above of=7, node distance=.5cm] {\includegraphics[width=1cm]{../figuras/licao02/carrinho2.png}};

\path[thick]

(larg) edge (1)
(1) edge (2)
(2) edge (3)
(3) edge (4)
(4) edge (5)
(5) edge (6)
% (6) edge (7)
;
\end{tikzpicture}
\end{center}
  

  Do item f), o carrinho de Lorenzo percorreu $\frac{6}{4}$ da distância percorrida pelo carrinho de Lucas. Para marcar seis quartos dessa unidade, divide-se a distância percorrida por Lucas em quartos e justapõe-se seis partes iguais a essa a partir do ponto de partida.

  Agora surge a dúvida das posições relativas de paradas dos carrinhos de Lorenzo e Heitor. Para isso é importante observar que dois quartos corresponde à metade da distância percorrida pelo carrinho de Lucas e, portanto, que seis quartos corresponde à três metades (três meios) dessa distância. Logo os carrinhos de Lorenzo e Heitor pararam na mesma posição.

  \hspace{-15mm} 

\begin{center}
\begin{tikzpicture}[every node/.style={scale=.7}]

\tikzstyle{circ}=[draw, circle, inner sep=.075cm, node distance=1.75cm, thick];

\node (larg) [circ, label=below:Largada] {} ;

\node (1) [circ, right of=larg, label=below:Matheus] {} node [above of=1, node distance=.5cm]{\includegraphics[width=1cm]{../figuras/licao02/carrinho2.png}};

\node (2) [circ, right of=1, label=below:Lucas] {} node [above of=2, node distance=.5cm]{\includegraphics[width=1cm]{../figuras/licao02/carrinho2.png}};

\node (3) [circ, right of=2, label={[below of=4, align=center, node distance=.7cm] Heitor e\\Lorenzo}] {} node [above of=3, node distance=.5cm] {\includegraphics[width=1cm]{../figuras/licao02/carrinho2.png}};

\node (4) [circ, right of=3, label=below:Rafael] {} node [above of=4, node distance=.5cm] {\includegraphics[width=1cm]{../figuras/licao02/carrinho2.png}};

\node (5) [circ, right of=4, label=below:{Enzo}] {} node [above of=5, node distance=.5cm] {\includegraphics[width=1cm]{../figuras/licao02/carrinho2.png}};

\node (6) [circ, right of=5, label=below:{Nicolas}] {} node [above of=6, node distance=.5cm] {\includegraphics[width=1cm]{../figuras/licao02/carrinho2.png}};

% \node (7) [circ, right of=6, label=below:Guilherme] {} node [above of=7, node distance=.5cm] {\includegraphics[width=1cm]{../figuras/licao02/carrinho2.png}};

\path[thick]

(larg) edge (1)
(1) edge (2)
(2) edge (3)
(3) edge (4)
(4) edge (5)
(5) edge (6)
% (6) edge (7)
;
\end{tikzpicture}
\end{center}


Do item g), o carrinho de Guilherme percorreu o dobro da distância percorrida pelo carrinho de Lucas. Para marcar no encarte corretamente é necessário observar que a posição de parada do carrinho de Rafael também corresponde ao dobro da distância percorrida pelo carrinho de Lucas. Para isso basta perceber que como duas metades da unidade corresponde à unidade, quatro metades dessa mesma unidade correspondem à duas unidades.

\hspace{-15mm} 

\begin{center}
\begin{tikzpicture}[every node/.style={scale=.7}]

\tikzstyle{circ}=[draw, circle, inner sep=.075cm, node distance=1.75cm, thick];

\node (larg) [circ, label=below:Largada] {} ;

\node (1) [circ, right of=larg, label=below:Matheus] {} node [above of=1, node distance=.5cm]{\includegraphics[width=1cm]{../figuras/licao02/carrinho2.png}};

\node (2) [circ, right of=1, label=below:Lucas] {} node [above of=2, node distance=.5cm]{\includegraphics[width=1cm]{../figuras/licao02/carrinho2.png}};

\node (3) [circ, right of=2, label={[below of=4, align=center, node distance=.7cm] Heitor e\\Lorenzo}] {} node [above of=3, node distance=.5cm] {\includegraphics[width=1cm]{../figuras/licao02/carrinho2.png}};

\node (4) [circ, right of=3, label={[below of=4, align=center, node distance=.7cm] Rafael e\\ Guilherme}] {} node [above of=4, node distance=.5cm] {\includegraphics[width=1cm]{../figuras/licao02/carrinho2.png}};

\node (5) [circ, right of=4, label=below:{Enzo}] {} node [above of=5, node distance=.5cm] {\includegraphics[width=1cm]{../figuras/licao02/carrinho2.png}};

\node (6) [circ, right of=5, label=below:{Nicolas}] {} node [above of=6, node distance=.5cm] {\includegraphics[width=1cm]{../figuras/licao02/carrinho2.png}};

% \node (7) [circ, right of=6, label=below:Guilherme] {} node [above of=7, node distance=.5cm] {\includegraphics[width=1cm]{../figuras/licao02/carrinho2.png}};

\path[thick]

(larg) edge (1)
(1) edge (2)
(2) edge (3)
(3) edge (4)
(4) edge (5)
(5) edge (6)
% (6) edge (7)
;
\end{tikzpicture}
\end{center}


Para se determinar a posição de parada do carrinho de Samuel é necessário identificar a posição que corresponde a $\frac{6}{3}$ da distância percorrida pelo carrinho de Lucas. Para fazer isso deve-se dividir essa unidade (distância percorrida pelo carrinho de Lucas) em três partes iguais e justapor 6 dessas partes a partir do ponto de partida. Mas para marcar corretamente essa posição no encarte é necessário comparar essa fração da unidade com as demais. Sabe-se que três terços correspondem a uma unidade, logo seis terços devem corresponder a duas unidades. Portanto, a posição de parada do carrinho de Samuel coincide com a posição de parada dos carrinhos de Guilherme e de Rafael.

\hspace{-15mm} 

\begin{center}


\begin{tikzpicture}[every node/.style={scale=.7}]

\tikzstyle{circ}=[draw, circle, inner sep=.075cm, node distance=1.75cm, thick];

\node (larg) [circ, label=below:Largada] {} ;

\node (1) [circ, right of=larg, label=below:Matheus] {} node [above of=1, node distance=.5cm]{\includegraphics[width=1cm]{../figuras/licao02/carrinho2.png}};

\node (2) [circ, right of=1, label=below:Lucas] {} node [above of=2, node distance=.5cm]{\includegraphics[width=1cm]{../figuras/licao02/carrinho2.png}};

\node (3) [circ, right of=2, label={[below of=4, align=center, node distance=.7cm] Heitor e\\Lorenzo}] {} node [above of=3, node distance=.5cm] {\includegraphics[width=1cm]{../figuras/licao02/carrinho2.png}};

\node (4) [circ, right of=3, label={[below of=4, align=center, node distance=.925cm] Rafael,\\Samuel e\\ Guilherme}] {} node [above of=4, node distance=.5cm] {\includegraphics[width=1cm]{../figuras/licao02/carrinho2.png}};

\node (5) [circ, right of=4, label=below:{Enzo}] {} node [above of=5, node distance=.5cm] {\includegraphics[width=1cm]{../figuras/licao02/carrinho2.png}};

\node (6) [circ, right of=5, label=below:{Nicolas}] {} node [above of=6, node distance=.5cm] {\includegraphics[width=1cm]{../figuras/licao02/carrinho2.png}};

% \node (7) [circ, right of=6, label=below:Guilherme] {} node [above of=7, node distance=.5cm] {\includegraphics[width=1cm]{../figuras/licao02/carrinho2.png}};

\path[thick]

(larg) edge (1)
(1) edge (2)
(2) edge (3)
(3) edge (4)
(4) edge (5)
(5) edge (6)
% (6) edge (7)
;
\end{tikzpicture}
\end{center}


Observe que os carrinhos de Rafael e Samuel pararam no mesmo lugar!
\end{solucao}

\begin{objetivos}[label=chap2-ativ13]{}{}

\begin{itemize} %s
    \item Reconhecer que uma mesma quantidade pode ser expressa por frações diferentes dependendo da unidade escolhida.
\end{itemize} %s
\end{objetivos}

\begin{orientacoes}

  \begin{itemize} %s
  \item Uma resposta possível é que dois (ou até os três) estudantes tenham errado em suas afirmações.    Muito provavelmente, nesse caso, os estudantes estão considerando a mesma unidade - bolos de mesmo tamanho. 
Como o objetivo da questão é levar os alunos a concluírem que unidades diferentes podem determinar a identificação de frações diferentes para uma mesma quantidade, cabe ao professor instigar essa discussão com os alunos. Espera-se que leve-os a considerar o caso de os bolos originais terem tamanhos diferentes. Isso pode ser feito com perguntas como, por exemplo: É possível uma tal situação acontecer estando os três amigos certos?

Será que os bolos eram iguais? Será que tinham o mesmo tamanho? 
\end{itemize}
\end{orientacoes}

\begin{solucao}{}{}
  Se os bolos tivessem tamanhos iguais, um quinto do bolo teria menor quantidade que um terço do bolo, que por sua vez corresponderia a menos bolo que metade. Como todos os estudantes tinham a mesma quantidade, os bolos eram diferentes.

  \end{solucao}


\begin{objetivos}[label=chap2-ativ14]{}{}
  \begin{itemize} %s
    \item       Perceber que uma mesma fração (no caso, $\frac{1}{2}$) de unidades diferentes pode resultar em quantidades diferentes.
\end{itemize} %s
\end{objetivos}

\begin{orientacoes}
  \begin{itemize} %s
    \item       Esta é uma atividade que o aluno pode fazer individualmente, mas é essencial que seja discutida com toda a turma.
    \item       Como fechamento da atividade, é importante enfatizar que  uma mesma fração de unidades diferentes pode resultar em quantidades diferentes. Sugere-se ressaltar aos estudantes que, no contexto ``frações de'', é fundamental saber a que o ``de'' se refere, isto é, qual é a unidade que está sendo considerada. Neste sentido, esta atividade está fortemente relacionada com a Atividade 8.
\end{itemize} %s

%  \noindent {\bf Classificações:}\vspace{.1cm}
% \begin{itemize} %s
%     \item       Heid et al.: Raciocínio: corroborar
%     \item       Nicely, Jr.: Nível 6: justificar
%     \item       UERJ: Analisar: levantar hipóteses
% \end{itemize} %s

\end{orientacoes}

\begin{solucao}{}{}
  José está certo se a pizza da qual comeu metade for maior do que a pizza da qual Ella comeu metade, como ilustra a figura a seguir.
  \begin{center}
   \begin{tikzpicture}[x=1mm,y=1mm]
    \draw (0,0) circle (4);
    \draw[fill=attention] (90:4) arc (90:270:4) --  cycle;
    \draw[fill=common, fill opacity=.3] (90:4) arc (90:-90:4) --  cycle;
    \node at (0,8) {Pizza de Ella};
    \begin{scope}[shift={(20,0)}]
    \draw (0,0) circle (8);
    \draw[fill=attention] (90:8) arc (90:270:8) --  cycle;
    \draw[fill=common, fill opacity=.3] (90:8) arc (90:-90:8) --  cycle;
    \node at (0,12) {Pizza de José};
    \end{scope}
   \end{tikzpicture}

  \end{center}

  \end{solucao}

%\newpage
\begin{objetivos}[label=chap2-ativ15]{}{}
\begin{itemize} %s
    \item Reconhecer que uma mesma quantidade pode ser expressa por frações diferentes dependendo da unidade escolhida.
    \item Utilizar linguagem simbólica para referir-se a uma fração~$\frac{a}{b}$.
\end{itemize} %s

\end{objetivos}

\begin{orientacoes}{}{}
  \begin{itemize} %s
  \item Recomenda-se que a atividade seja desenvolvida em grupos de 3 a 5 alunos.
    \item Recomenda-se que os itens sejam propostos em blocos (de três em três, por exemplo) intercalados com a correção. Tendo em vista que se o estudante não atingiu os objetivos da atividade nos primeiros itens, ele provavelmente não conseguirá fazê-lo nas seguintes sem a intervenção do professor.
    \item As diversas soluções apresentadas devem ser discutidas com a
turma inteira. É possível que os alunos utilizem frações equivalentes como
resposta para um mesmo item. Por exemplo, no item f), as frações $\frac{3}{6}$ e $\frac{1}{2}$ são respostas corretas. Nesses casos, dê a oportunidade para que cada aluno explique como chegou a sua resposta. Os alunos perceberão que uma mesma quantidade pode ser descrita por frações com numeradores e denominadores diferentes. Isso vai prepará-los para o assunto frações equivalentes, que será tratado na Lição 4.
    \item Observe que, no contexto ``frações de'', é fundamental saber a que o ``de'' se refere, isto é, qual é a unidade que está sendo considerada. Assim, no final da atividade, é importante enfatizar para os alunos que uma mesma quantidade pode ser descrita por frações diferentes com unidades diferentes.
\end{itemize} %s

%  \vspace{.1cm}

%  \noindent {\bf Classificações:}\vspace{.1cm}
% \begin{itemize} %s
%     \item       Heid et al.: Conceito: identificar; Produto: gerar
%     \item       Nicely, Jr.: Nível 1: reconhecer; Nível 5: converter (simbolizar)
%     \item       UERJ: Interpretar: discriminar
% \end{itemize} %s
% \end{multicols}

\end{orientacoes}
\end{multicols}

\newpage

\begin{solucao}{}{}

\noindent\begin{tabular}{m{.12\textwidth}m{.12\textwidth}m{.12\textwidth}m{.14\textwidth}m{.14\textwidth}m{.14\textwidth}}
  a) $\dfrac{1}{2}$. &  b) $\dfrac{1}{4}$. & c) $\dfrac{1}{6}$. & d) $\dfrac{3}{2}$. &  e) $\dfrac{3}{4}$. &  f) $\dfrac{1}{2}$ ou $\dfrac{3}{6}$.   \\
\\
g) $\dfrac{5}{2}$. &  h) $\dfrac{5}{4}$. & i) $\dfrac{5}{6}$. & j) $3$ ou $\dfrac{6}{2}$.
    &  k) $\dfrac{3}{2}$ ou $\dfrac{6}{4}$. &  l) $1$ ou $\dfrac{3}{3}$ ou $\dfrac{6}{6}$.
\end{tabular} %s

\end{solucao}
\null
\vfill

% \begin{multicols}{2}
    
  \begin{objetivos}[label=chap2-ativ16]{}{}
  \begin{itemize} %s
    \item Diferenciar ``a divisão da unidade em cinco partes quaisquer'' da ``divisão da unidade em cinco partes iguais''.
\end{itemize} %s
\end{objetivos}

\begin{orientacoes}
\begin{itemize} %s
    \item       Esta é uma atividade que o aluno pode fazer individualmente, mas é essencial que seja discutida com toda a turma.
    \item       No final da atividade, é importante salientar que o fato de  uma figura estar divida em 5 partes e 3 delas estarem pintadas de vermelho,       {\bf não necessariamente implica}       que a região pintada é       $\frac{3}{5}$ da figura.
    \item       O tipo de situação descrita na atividade é um equívoco comum entre os alunos, isto é, eles equivocadamente contam partes sem o cuidado de verificar se as partes nas quais a unidade está dividida correspondem a uma mesma quantidade.
\end{itemize} %s

%  \noindent {\bf Classificações:}\vspace{.1cm}
% \begin{itemize} %s
%     \item       Heid et al.: Raciocínio: corroborar
%     \item       Nicely, Jr.: Nível 9: avaliar
%     \item       UERJ: Avaliar: julgar
% \end{itemize} %s

\end{orientacoes}

\begin{solucao}{}{}
Miguel está equivocado: a região pintada da figura   {\bf não}   corresponde a   $\frac{3}{5}$ da figura porque a figura não está dividida em 5 partes ``iguais'', ou seja, a figura não está ``dividida em partes iguais'' em 5 partes para que as 3 partes pintadas correspondam a   $\frac{3}{5}$ da mesma. Outra justificativa possível é: a parte pintada é formada por 3 partes iguais,porém,  justapondo-se 5 cópias de uma destas partes, não é possível recompor a figura inteira, logo, a parte pintada não é $\frac{3}{5}$ da figura.

\end{solucao}

\newpage
\begin{multicols}{2}
  
\begin{objetivos}[label=chap2-ativ17]{}{}
  \begin{itemize} %s
    \item Perceber que, se uma unidade foi equiparticionada em $n + m$ partes iguais, das quais $n$ foram pintadas, então $\frac{n}{m}$ {\bf não especifica} a fração da unidade que foi pintada.
\end{itemize} %s
\end{objetivos}

\begin{orientacoes}
\begin{itemize} %s
    \item       Esta é uma atividade que o aluno pode fazer individualmente, mas é essencial que seja discutida com toda a turma.
    \item       O tipo de situação descrita na atividade destaca um equívoco comum entre os alunos. Assim, esta atividade é uma oportunidade para reforçar os papéis do denominador e do numerador na notação simbólica matemática para frações: o denominador especifica o número de partes iguais em que a unidade foi dividida e o numerador especifica o número de cópias que foram tomadas de uma destas partes.
      \item Considere perguntar aos alunos que fração a parte vermelha é da parte azul.

\end{itemize} %s
%  \noindent {\bf Classificações:}\vspace{.1cm}
% \begin{itemize} %s
%     \item       Heid et al.: Raciocínio: corroborar
%     \item       Nicely, Jr.: Nível 9: avaliar
%     \item       UERJ: Avaliar: julgar
% \end{itemize} %s

\end{orientacoes}


\begin{solucao}{}{}
  A parte pintada de vermelho   {\bf não}   corresponde a   $\frac{3}{4}$ da figura. Ela corresponde a $\frac{3}{7}$ da figura, pois a figura foi dividida em   $7$ partes iguais das quais $3$ foram pintadas.
\end{solucao}

\begin{objetivos}[label=chap2-ativ18]{}{}
\begin{itemize} %s
    \item       Perceber a importância da explicitação unidade na representação de quantidades.
\end{itemize} %s
\end{objetivos}

\begin{orientacoes}
\begin{itemize} %s
    \item       Esta é uma atividade que o aluno pode fazer individualmente, mas é essencial que seja discutida com toda a turma.
    \item       Recomenda-se que os itens da atividade sejam feitos e corrigidos um a um, de forma a permitir que um aluno que tenha errado um item possa acertar o seguinte.
    \item       O fato de a unidade não estar explicitada, torna ambígua a questão. É importante que os alunos percebam que, por exemplo, no item a), se a unidade considerada for um dos hexágonos, a fração correspondente à região em vermelho é $\frac{1}{2}$. No entanto, se forem os dois hexágonos, é $\frac{1}{4}$.
    \item       No final de cada item da atividade, é importante enfatizar para seus alunos que uma mesma quantidade pode ser expressa por frações diferentes dependendo da unidade escolhida. Observe para eles que, no contexto       ``frações de'', é fundamental saber a que o       ``de''     se refere, isto é, qual é a unidade que está sendo considerada. Neste sentido, esta atividade está fortemente relacionada com as Atividades 8 e 13. Ela também é uma preparação para a Atividade 18, em que a mesma questão é posta, mas agora com um modelo mais comumente usado e, portanto, mais resistente à reflexão que se deseja estabelecer.
\end{itemize} %s
\end{orientacoes}
%  \noindent {\bf Classificações:}\vspace{.1cm}
% \begin{itemize} %s
%     \item       Heid et al.: Raciocínio: corroborar
%     \item       Nicely, Jr.: Nível 9: avaliar
%     \item       UERJ: Avaliar: julgar
% \end{itemize} %s

\def \tripinha{ (30:4) -- (90:4) -- (150:4)--(210:4)--(270:4)--(330:4) [shift={({4*sqrt(3)},0)}] --(270:4) -- (330:4) -- (30:4) -- (90:4)--(150:4)--cycle}


\def \tripa{ (30:4) -- (90:4) -- (150:4)--(210:4)--(270:4)--(330:4) [shift={({4*sqrt(3)},0)}] --(270:4) -- (330:4) [shift={({4*sqrt(3)},0)}]--  (270:4) -- (330:4) -- (30:4) -- (90:4)--(150:4) [shift={({-4*sqrt(3)},0)}] -- (90:4) -- (150:4)--cycle;}

\def \tripalonga{ (30:4) -- (90:4) -- (150:4)--(210:4)--(270:4)--(330:4) [shift={({4*sqrt(3)},0)}] --(270:4) -- (330:4) [shift={({4*sqrt(3)},0)}] --(270:4) -- (330:4)[shift={({4*sqrt(3)},0)}] --(270:4) -- (330:4) [shift={({4*sqrt(3)},0)}]--  (270:4) -- (330:4) -- (30:4) -- (90:4)--(150:4) [shift={({-4*sqrt(3)},0)}] -- (90:4) -- (150:4)[shift={({-4*sqrt(3)},0)}] -- (90:4) -- (150:4) [shift={({-4*sqrt(3)},0)}] -- (90:4) -- (150:4)--cycle;}


\begin{solucao}{}{}
\begin{enumerate} [\quad a)] %s
    \item       A região em vermelho pode representar       $\frac{1}{2}$ ou       $\frac{1}{4}$ dependendo da unidade, que não foi explicitada no enunciado. Se, por exemplo, a unidade for
    \begin{tikzpicture}[x=1mm,y=1mm]
     \draw[fill=common, fill opacity=.3] (0:4) -- (60:4)--(120:4)-- (180:4)--(240:4)--(300:4)--cycle;
    \end{tikzpicture}
 então a região pintada de vermelho em  \begin{tikzpicture}[x=1mm,y=1mm]
 \draw[fill=common, fill opacity=.3] \tripinha;
\begin{scope}
 \clip \tripinha;
\draw[fill=attention] (-4,-4) rectangle (0,4);
\end{scope}
\end{tikzpicture}
é   $\frac{1}{2}$ dessa unidade. Por outro lado,  se a unidade for   \begin{tikzpicture}[x=1mm,y=1mm]
                                                                        \draw[fill=common, fill opacity=.3] \tripinha;
                                                                       \end{tikzpicture}
   então a região pintada de vermelho em  \begin{tikzpicture}[x=1mm,y=1mm]
 \draw[fill=common, fill opacity=.3] \tripinha;
\begin{scope}
 \clip \tripinha;
\draw[fill=attention] (-4,-4) rectangle (0,4);
\end{scope}
\end{tikzpicture}    é   $\frac{1}{4}$ dessa unidade.

    \item       A região em vermelho pode representar       $\frac{1}{2}$ ou       $\frac{3}{2}$ dependendo da unidade, que não foi explicitada no enunciado. Se, por exemplo, a unidade for \begin{tikzpicture}[x=1mm,y=1mm] \draw[fill=common, fill opacity=.3] \tripa  \end{tikzpicture} então a região pintada de vermelho em \begin{tikzpicture}[x=1mm,y=1mm]
 \draw[fill=common, fill opacity=.3] \tripa;
\begin{scope}
 \clip \tripa;
\draw[fill=attention] (-4,-4) rectangle ({4*sqrt(3)},4);
\end{scope}
\end{tikzpicture}
   é   $\frac{1}{2}$ dessa unidade. Por outro lado,  se a unidade for  \begin{tikzpicture}[x=1mm,y=1mm]
     \draw[fill=common, fill opacity=.3] (0:4) -- (60:4)--(120:4)-- (180:4)--(240:4)--(300:4)--cycle;
    \end{tikzpicture}  então a região pintada de vermelho em
    \begin{tikzpicture}[x=1mm,y=1mm]
    \draw[fill=common, fill opacity=.3] \tripa;
    \begin{scope}
    \clip \tripa;
    \draw[fill=attention] (-4,-4) rectangle ({4*sqrt(3)},4);
    \end{scope}
    \end{tikzpicture}    é   $\frac{3}{2}$ dessa unidade.

    \item       A região em vermelho pode representar       $\frac{3}{5}$ ou       $3$ dependendo da unidade, que não foi explicitada no enunciado. Se, por exemplo, a unidade for \begin{tikzpicture}[x=1mm,y=1mm] \draw[fill=common, fill opacity=.3] \tripalonga  \end{tikzpicture} então a região pintada de vermelho em \begin{tikzpicture}[x=1mm,y=1mm]
 \draw[fill=common, fill opacity=.3] \tripalonga;
\begin{scope}
 \clip \tripalonga;
  \draw[fill=attention] (-4,-4) rectangle ({10*sqrt(3)},4);
\end{scope}
\end{tikzpicture}
  é   $\frac{3}{5}$ dessa unidade. Por outro lado,  se a unidade for \begin{tikzpicture}[x=1mm,y=1mm]
     \draw[fill=common, fill opacity=.3] (0:4) -- (60:4)--(120:4)-- (180:4)--(240:4)--(300:4)--cycle;
    \end{tikzpicture} então a região pintada de vermelho em \begin{tikzpicture}[x=1mm,y=1mm]
 \draw[fill=common, fill opacity=.3] \tripalonga;
\begin{scope}
 \clip \tripalonga;
  \draw[fill=attention] (-4,-4) rectangle ({10*sqrt(3)},4);
\end{scope}
\end{tikzpicture} é   $3$ dessa unidade.
\end{enumerate} %s
\end{solucao}


\begin{objetivos}[label=chap2-ativ19]{}{}
\begin{itemize} %s
    \item       Perceber a importância da unidade na representação de quantidades.
\end{itemize} %s

\end{objetivos}

\begin{orientacoes}
\begin{itemize} %s
    \item       Esta é uma atividade que o aluno pode fazer individualmente, mas é essencial que seja discutida com toda a turma.
    \item       No final da atividade, é importante enfatizar para seus alunos a propriedade matemática que esta atividade quer destacar, ou seja, que uma mesma quantidade pode ser expressa por frações diferentes dependendo da unidade escolhida. Observe para eles que, no contexto       ``frações de'', é fundamental saber a que o       ``de''     se refere, isto é, qual é a unidade que está sendo considerada. Neste sentido, esta atividade está fortemente relacionada com as Atividades 8 e 13.
\end{itemize} %s

 % \noindent {\bf Classificações:}\vspace{.1cm}
% \begin{itemize} %s
%     \item       Heid et al.: Raciocínio: corroborar
%     \item       Nicely, Jr.: Nível 9: avaliar
%     \item       UERJ: Avaliar: julgar
% \end{itemize} %s

\end{orientacoes}

\begin{solucao}{}{}
  As afirmações de Júlia, Davi e Laura estão incompletas, pois eles não informaram a   {\bf unidade}. De fato, dependendo da escolha da unidade, cada um deles pode estar certo e os demais errados. Por exemplo, se a unidade for
\begin{center}
  \begin{tikzpicture}[x=1mm,y=1mm]
 \draw[fill=common, fill opacity=.3, scale=4] (0,0) rectangle (5,3);
\end{tikzpicture}
\end{center}
então a parte pintada de vermelho em
\begin{center}
\begin{tikzpicture}[x=1mm,y=1mm]
 \draw[fill=attention, scale=4] (0,0) rectangle (3,3);
 \draw[fill=common, fill opacity=.3, scale=4] (3,0) rectangle (5,3);
 \foreach \x in {1,2,4} \draw[scale=4] (\x,0) -- (\x,3);
\end{tikzpicture}
\end{center}
de fato corresponde a   $\frac{3}{5}$ desta unidade, de modo que, nesta situação, Júlia está certa e David e Laura estão errados. Contudo, se a unidade for
\begin{center}
\begin{tikzpicture}[x=1mm,y=1mm]
 \draw[fill=common, fill opacity=.3, scale=4] (0,0) rectangle (2,3);
\end{tikzpicture}
\end{center}
então a parte pintada de vermelho em
\begin{center}
\begin{tikzpicture}[x=1mm,y=1mm]
 \draw[fill=attention, scale=4] (0,0) rectangle (3,3);
 \draw[fill=common, fill opacity=.3, scale=4] (3,0) rectangle (5,3);
 \foreach \x in {1,2,4} \draw[scale=4] (\x,0) -- (\x,3);
\end{tikzpicture}
\end{center}
  corresponde a   $\frac{3}{2}$ desta unidade,  de modo que, nesta situação, David está certo e Júlia e Laura estão errados. Finalmente, se a unidade for
\begin{center}
\begin{tikzpicture}[x=1mm,y=1mm]
 \draw[fill=common, fill opacity=.3, scale=4] (0,0) rectangle (1,3);
\end{tikzpicture}
\end{center}
então a parte pintada de vermelho em
\begin{center}
\begin{tikzpicture}[x=1mm,y=1mm]
 \draw[fill=attention, scale=4] (0,0) rectangle (3,3);
 \draw[fill=common, fill opacity=.3, scale=4] (3,0) rectangle (5,3);
 \foreach \x in {1,2,4} \draw[scale=4] (\x,0) -- (\x,3);
\end{tikzpicture}
\end{center}
  corresponde a   $3$ desta unidade e, neste caso, Laura está certa e David e Júlia estão errados.
\end{solucao}


\begin{objetivos}[label=chap2-ativ20]{}{}

\begin{itemize} %s
    \item       Relembrar divisão com resto (ou divisão euclidiana).
    \item       Selecionar, dentro de uma situação plausível do dia a dia, dados relevantes para resolver um problema.
\end{itemize} %s
\vspace{.1cm}
\end{objetivos}

\begin{orientacoes}

\begin{itemize} %s
    \item       A atividade deve ser conduzida de forma a chegar-se na divisão euclidiana. Ou seja, o aluno pode começar montando as pizzas. Recomenda-se que os alunos tenham à mão o material concreto: fatias de pizza cortadas em papel ou em E.V.A..
    \item       É possível que os alunos resolvam o item a) a partir da divisão euclidiana, efetuando a divisão de 38 por 8: $38 = 4 \times 8 + 6$. Se esse for o caso, recomenda-se que o professor, destaque a informação associada a cada um dos números na expressão. Em particular, o       ``resto'', que identifica uma quantidade menor do que uma pizza (resto 6 indica 6 fatias, que é menor do que uma pizza, uma vez que cada pizza tem 8 fatias).
    \item       Para responder ao item b), o aluno deve reconhecer que cada fatia é igual a       $\frac{1}{8}$ da pizza. Portanto, a quantidade total de pizza consumida pelos amigos pode ser expressa como       $\frac{38}{8}$ de uma pizza. Cabe destacar que essa fração corresponde a 4 pizzas mais $\frac{6}{8}$ de uma pizza.
    \item        Observe que, neste contexto, o resto, que é um número inteiro e indica o número de fatias, também pode ser expresso por meio de uma fração da unidade pizza:       $\frac{6}{8}$ de uma pizza.
    \item       Faz parte da atividade a tarefa de selecionar dados relevantes para o problema, o que a torna um tanto complexa, por isso é a última Atividade da Lição 2. Para os itens a) e b), a quantidade de amigos, 7, é irrelevante. No entanto, é relevante para o item c).
    \item       A atividade tem também o objetivo de evidenciar que, no cotidiano, nem toda partição é uma equipartição: 38 fatias de pizza para 7 amigos é um exemplo.
\end{itemize} %s
\end{orientacoes}

\begin{solucao}{}{}

\begin{enumerate} [\quad a)] %s
    \item       A solução corresponde ao quociente da divisão euclidiana de 38 por 8, ou seja, 4
    \item       Compreendendo que cada fatia é       $\frac{1}{8}$ de uma pizza: 4 pizzas e       $\frac{6}{8}$ ou       $\frac{38}{8}$.
    \item       A divisão euclidiana de 38 por 7 fornece um resto diferente de zero, o que indica que não é possível que todos os amigos tenham comido o mesmo número de fatias de pizza.
\end{enumerate} %s
\end{solucao}


\end{multicols}

%%% Local Variables: 
%%% mode: latex
%%% TeX-master: "livro_professor_completo.tex"
%%% End: 
